% universal settings
\documentclass[smalldemyvopaper,11pt,twoside,onecolumn,openright,extrafontsizes]{memoir}
\usepackage[utf8x]{inputenc}
\usepackage[T1]{fontenc}
\usepackage[osf]{Alegreya,AlegreyaSans}

% PACKAGE DEFINITION
% typographical packages
\usepackage{microtype} % for micro-typographical adjustments
\usepackage{setspace} % for line spacing
\usepackage{lettrine} % for drop caps and awesome chapter beginnings
\usepackage{titlesec} % for manipulation of chapter titles

% for placeholder text
\usepackage{lipsum} % to generate Lorem Ipsum

% other
\usepackage{calc}
\usepackage{hologo}
\usepackage[hidelinks]{hyperref}
%\usepackage{showframe}
\usepackage{soul}

% PHYSICAL DOCUMENT SETUP
% media settings
\setstocksize{8.5in}{5.675in}
\settrimmedsize{8.5in}{5.5in}{*}
\setbinding{0.175in}
\setlrmarginsandblock{0.611in}{1.222in}{*}
\setulmarginsandblock{0.722in}{1.545in}{*}

% defining the title and the author
%\title{\LaTeX{} ePub Template}
%\title{\textsc{How I Started to Love {\fontfamily{cmr}\selectfont\LaTeX{}}}}
\title{Komputer}
\author{Naru Aika}
\newcommand{\ISBN}{0-000-00000-2}
\newcommand{\press}{}

% custom second title page
\makeatletter
\newcommand*\halftitlepage{\begingroup % Misericords, T&H p 153
  \setlength\drop{0.1\textheight}
  \begin{center}
  \vspace*{\drop}
  \rule{\textwidth}{0in}\par
  {\Large\textsc\thetitle\par}
  \rule{\textwidth}{0in}\par
  \vfill
  \end{center}
\endgroup}
\makeatother

% custom title page
\thispagestyle{empty}
\makeatletter
\newlength\drop{}
\newcommand*\titleM{\begingroup % Misericords, T&H p 153
  \setlength\drop{0.15\textheight}
  \begin{center}
  \vspace*{\drop}
  \rule{\textwidth}{0in}\par
  {\HUGE\textsc\thetitle\par}
  \rule{\textwidth}{0in}\par
  {\Large\textit\theauthor\par}
  \vfill
  {\Large\scshape\press}
  \end{center}
\endgroup}
\makeatother

% chapter title manipulation
% padding with zero
\renewcommand*\thechapter{\ifnum\value{chapter}<10 0\fi\arabic{chapter}}
% chapter title display
\titleformat
{\chapter}
[display]
{\normalfont\scshape\huge}
{\HUGE\thechapter\centering}
{0pt}
{\vspace{18pt}\centering}[\vspace{42pt}]

% typographical settings for the body text
\setlength{\parskip}{0em}
\linespread{1.09}

% HEADER AND FOOTER MANIPULATION
  % for normal pages
  \nouppercaseheads{}
  \headsep = 0.16in
  \makepagestyle{mystyle}
  \setlength{\headwidth}{\dimexpr\textwidth+\marginparsep+\marginparwidth\relax}
  \makerunningwidth{mystyle}{\headwidth}
  \makeevenhead{mystyle}{}{\textsf{\scriptsize\scshape\thetitle}}{}
  \makeoddhead{mystyle}{}{\textsf{\scriptsize\scshape\leftmark}}{}
  \makeevenfoot{mystyle}{}{\textsf{\scriptsize\thepage}}{}
  \makeoddfoot{mystyle}{}{\textsf{\scriptsize\thepage}}{}
  \makeatletter
  \makepsmarks{mystyle}{%
  \createmark{chapter}{left}{nonumber}{\@chapapp\ }{.\ }}
  \makeatother
  % for pages where chapters begin
  \makepagestyle{plain}
  \makerunningwidth{plain}{\headwidth}
  \makeevenfoot{plain}{}{}{}
  \makeoddfoot{plain}{}{}{}
  \pagestyle{mystyle}
% END HEADER AND FOOTER MANIPULATION

% table of contents customisation
\renewcommand\contentsname{\normalfont\scshape Daftar Isi}
\renewcommand\cftchapterfont{\normalfont}
\renewcommand{\cftchapterpagefont}{\normalfont}
\renewcommand{\printtoctitle}{\centering\Huge}

% layout check and fix
\checkandfixthelayout{}
% \fixpdflayout

% custom
\newcommand\separator{
  \begin{center}
    \(\ast~\ast~\ast\)
  \end{center}
}

% BEGIN THE DOCUMENT
\begin{document}
\pagestyle{empty}
% the half title page
% \halftitlepage
% \cleardoublepage
% the title page
\titleM{}
\clearpage
% copyright page
% \noindent{\small{This novel is entirely a work of fiction. The names, characters and incidents portrayed in it are the product of the author's imagination. Any resemblance to actual persons, living or dead, or events or localities is entirely coincidental.\par\vfill\noindent Paperback Edition\space\today\\ISBN\space\ISBN\\\copyright\space\theauthor. All rights reserved.\par\vfill\noindent\theauthor\space asserts the moral right to be identified as the author of this work. All rights reserved in all media. No part of this publication may be reproduced, stored in a retrieval system, or transmitted, in any form, or by any means, electronic, mechanical, photocopying, recording or otherwise, without the prior written permission of the author and/or the publisher.\par}}
% \clearpage

% dedication
% \begin{center}
% \itshape{\noindent{Untuk para penyuka novel picisan.}}
% \end{center}

% begin front matter
\frontmatter{}
\pagestyle{mystyle}
% preface
% \chapter*{Kata Pengantar}
% \lipsum[100-104]
% acknowledgements
% \chapter*{Ucapan Terima Kasih}
% \lipsum[1-9]
% table of contents
% \clearpage
% \tableofcontents*

% begin main matter
\mainmatter{}

% genre
% misteri, fiksi ilmiah, prosa ungu

% sinopsis
% cerita fiksi ini mengisahkan tentang seorang yang telah  menjadi hikikomori selama lima belas tahun semenjak kematian sahabat paling karibnya. Suatu hari, dia mengiyakan penetapan sebuah janji. Dari pertemuan itu, dia bertekad untuk tidak akan membiarkan orang lain mengulangi kesalahan-kesalahan hidupnya dengan mempelajari ilmu komputer.

% penokohan
% - protagonis:
%   - aku/egg: seorang hikikomori yang suka berkomputer ria
%   - isha: seorang senator bijak yang mati dengan janggal
%   - asse: seorang penyuka misteri yang selalu membawa-bawa surya kanta di saku jas tebalnya
% - antagonis:
%   - lee: mantan anggota perwira yang pikirannya sulit ditebak, kakak kelas yang telah mengetahui kebenaran tentang pembunuhan isha
%   - patricia: adik isha yang telah membunuh kakaknya akibat masalah kesehatan
% - tritagonis:
%   - ara: seorang kawan yang canggung yang suka bercanda kelewatan
%   - karin: si pemurung sahabat dekat lee, hampir menjadi hikikomori namun terselamatkan oleh lee
%   - rei: seseorang misterius, kawan lama isha yang membantu menyingkap misteri kematian isha; sebetulnya ia bagian dari kelompok egg, namun hanya egg dan isha yang tahu

% gagasan cerita
% - apa saja yang bisa dilakukan dengan komputer? https://sites.google.com/site/liptonyunz/home/reason-of-why-people-use-computer
% - apa yang membuat komputer itu komputer? https://www.youtube.com/watch?v=mCq8-xTH7jA
% - bagaimana komputer bekerja? https://www.youtube.com/playlist?list=PLzdnOPI1iJNcsRwJhvksEo1tJqjIqWbN-
% - bagaimana internet bekerja? https://www.youtube.com/playlist?list=PLzdnOPI1iJNfMRZm5DDxco3UdsFegvuB7
% - bagaimana memilih komputer yang sesuai kebutuhan?
% - bagaimana mengoperasikan fungsi-fungsi dasar komputer?
% - bagaimana kecerdasan artifisial bekerja? https://www.youtube.com/watch?v=Ok-xpKjKp2g
% - bagaimana jaringan saraf tiruan bekerja? https://www.youtube.com/watch?v=JrXazCEACVo
% - bagaimana computer vision bekerja? https://www.youtube.com/watch?v=2hXG8v8p0KM
% - bagaimana mempersiapkan karir? lihat diagram ikigai
% - bagaimana membagi kebaikan?

\chapter*{Prolog}

\hyphenation{ke-hi-lang-an ke-ge-lap-an pa-ling me-nyi-sa-kan wa-tak per-bin-cang-kan be-la-kang ber-de-ring me-nyi-pit-kan men-de-ngar me-nyem-bu-nyi-kan da-lam apa-kah meng-he-la mun-cul aneh-nya meng-kha-wa-tir-kan ka-sih ke-sa-yang-an ter-se-nyum ka-ta-kan sa-ngat de-ngan meng-e-nak-kan ber-ma-in meng-geng-gam ba-sah ber-ki-sah me-ra-ih-nya makh-luk ber-bu-at me-nga-ju-kan meng-hi-rup akhir-nya ber-tong-kat per-ka-bung-an di-pra-ki-ra-kan mem-per-de-ngar-kan-nya meng-i-ngat ka-me-ra leng-ku-ngan meng-er-ti me-ni-kung pe-la-yan ber-ha-sil tang-kas pu-tih-nya me-mo-hon ber-ke-li-ling me-mak-sa pe-kan mem-per-ta-nya-kan}

Malam tak berangin itu, aku terbaring kosong bagai onggokan batang padi yang mengering di atas rerumputan tak bertuan di tepi jalan setapak yang berlubang; sebentar-sebentar memandang jauh ke langit yang terbentang luas pada hari kesatu. Terbenam ke dalam kesusahan hati yang berliku-liku tak menentu dari kumpulan kata tak berkesudahan yang menghunjam kencang ke dalam relung hatiku. ``Daripada mendengarkan banyak omong kosong,'' pikirku, ``lebih baik kulihat saja sendiri berkeliling sambil memohon secarik uraian terang yang cukup nyaring untuk menampik sangkaan tak segan manusia tentang diriku yang konon telah kecanduan mengonsumsi gadget.'' Namun tanpa disadari di kala itu aku justru masih menaruh tatapan erat ke dalam kelir-kelir menganjur dari layar gadget dan mengakhiri kalimat ini sambil menggerutu. ``Bah, betapa menyedihkannya aku ini,'' gumamku, ``Baiklah, kalau begitu, sebaliknya saja, mengapa orang-orang---'' (namun aku kurang senang untuk mempertanyakan diriku sendiri, sehingga tersebutlah \textit{orang-orang} menggantikan \textit{daku}) ``---sampai ketagihan, ya?'' Dalam benakku muncul sejumlah alasan dari ujaran-ujaran para tetua yang tak sampai lagi terdengar oleh muda mudi dalam satu dekade terakhir ini. ``Misalnya saja,'' batinku, ``dengan ponsel terhubung ke jejaring, orang-orang bisa bertukar pesan dengan siapa pun di seluruh penjuru dunia tanpa perlu memikirkan:

\textit{hari bulan kapan sampainya};

\textit{berapa koin ongkos kirimnya};

\textit{akan kepada siapa sampainya};

\textit{berapa hari banyak liburnya};

\textit{ke mana dikumpulkannya}.''

Sesudah menulis demikian, aku menambahkan, ``Anehnya, di pihak yang sama, orang-orang malah mulai mengkhawatirkan:

\textit{balasannya kapan datangnya};

\textit{berapa orang dikirimnya};

\textit{dengan siapa pengirimnya};

\textit{sinyalnya kapan kukuhnya};

\textit{ke mana dicadangkannya}.''

Tiba-tiba ponselku berdering ketibaan malam kian melarut, memekik di tengah kesunyian riang-riang yang sudah pulas. Dalam sekejap tercabut dari penguasaan lengahku untuk beroleh sebuah arloji tak bermerek dari pasar jengek yang terbungkus rapi di dalam kotaknya. Aku lupa bila sedang ditemani seseorang hampir sebaya saat itu. Dia menyipitkan matanya yang jelita sembari tersenyum memiringkan kepalanya lalu berkata, ``Selamat ulang tahun, Egg! Dua puluh tahun ini sungguh menyenangkan!'' biarpun ia belum setua itu. Entah apalah kehendaknya, seakan-akan kami berdua sama-sama memangku memori episodik dalam rentang waktu sedemikian panjangnya. Dengan keanggunannya tak bercela, dia mengajukan tangan mungilnya ke arahku, ``Ayo?'' Namun kuurungkan maksud untuk meraihnya seraya merunduk menyimpan gugup riuh nadiku dan menolaknya lembut, ``Tidak perlu. Terima kasih, ya.'' Dia pun menariknya kembali, mengembungkan pipi merahnya, merajuk tanpa kata, lantas bergegas menjauhiku. Enggan-enggan, aku bangkit menyiuk tidak berdaya lalu berjalan merapah---sembari memperhatikan lekat-lekat punggungnya dekat-dekat---menuju ke sekumpulan makhluk ijtimaiah yang sedang duduk-duduk melingkari api unggun sementara keasyikan berkisah kasih tentang kehidupan sekolahan dulu.

Berulang tahun memang tidak pernah menyenangkan, semuanya mengacungkan tangannya tanda setuju. Betapa tidak, kebanyakan mereka yang berumah di sekitarku lekas sekali mengeluarkan cek demi memestakan hiruk pikuk sisa hidupnya. Padahal boleh jadi besok atau lusa, orang yang diberinya selamat padam terkulai di atas usungan. Oleh karena itu, perkumpulan ini tidak dimaksudkan untuk merayakan sesuatu pun, meskipun kelihatannya masih bukan begitu. ``Biarlah saja, toh tidak ada untungnya juga bagiku mencampuri urusan semacam ini,'' pikirku. Sepatutnya aku lebih peduli terhadap gerak tawanya yang merdu dalam kesukaan perjamuan kecil-kecilan ini. Rasanya tidak pernah aku berbuat sedikit pun memikirkan perihal kehilangan hal itu. ``Jangan sampai, jangan sampai,'' ulangku. Lalu hilang membisu dalam bingkaian seroja.

\chapter{Saksi Bisu}

Di suatu pagi nan cerah---seusai hujan tak berjeda meng guyur di antara lenyapnya dua matahari di ufuk barat; membanjiri seperdua pemukiman di kota ini yang tidak memiliki drainase yang mumpuni---aku berhimpun kembali bersama teman-teman di sebuah kafetaria untuk sebuah janji yang tertunda. Sebagai seorang aku, jelas aku telah datang paling dahulu dengan setelan kasual yang kusut tak sempat kusterika pagi ini. Sembari duduk bertongkat lutut selama enggang menggeram di sebuah pojokan, menikmati sepiring kentang goreng yang baru saja dimasak, sesekali menghirup secangkir kopi hitam nan pahit, sekilas aku teringat sebuah peristiwa lima belas tahun silam saat seorang sahabat paling karib telah memaksa kami membuat rencana untuk datang kembali ke sini segera sesudah vaksinasi diumumkan. Beberapa pekan setelahnya, aku berbisik di dekat sahabatku yang rupawan tutur katanya itu, ``Seseorang yang membuat janji tidak seharusnya mati lebih dulu kan?'' Dia hanya diam tak berkutik. Pada akhirnya janji hanya sebatas janji; kami biarkan ia melanggar sendiri ucapan kesediaan yang telah terpatri sekian lama. Pandemi berkepanjangan ini sudah membuat sedu sedan rakyat jelata nestapa dalam perkabungan; menciutkan tekad mereka untuk tidak pergi keluar rumah. Masa depan tidak mungkin diketahui. Namun untuk diprakirakan, itu barangkali bisa terjadi. Tapi hanya jika sebab musababnya tidak saru dan bisa ditelaah. Umpamanya badan klimatologi yang hobinya meramal itu, terus memperdengarkannya melalui siaran radio, televisi, dan laman resmi pemerintahan sipil.

Kiranya seberapa nyata kerinduanku terhadap teman-temanku? Dan sebaliknya? Sememangnya, perkara abstrak tidak dapat diukur. Sebagaimana rasa pahit kopi ini relatif bagi siapa yang menyeruputnya. Seharusnya saja menyajikan kopi ini masih pahit; membiarkanku untuk mengatur sisanya. Cepat-cepat aku mengambil potret semangkuk kecil bungkalan gula batu dengan kamera saku lalu mengirimnya ke alat cetak jauh di kamar bacaku. Di era yang serba canggih ini, aku termenung, akankah nanti tempat ini dapat mengingat kesukaan setiap pelanggannya dalam menakar gula, lalu menerapkannya ke dalam kudapan yang dipesan? Itu mungkin saja akan membuat bersuka hati anak-anak perempuan yang melanggan tiap Sabtu.

Di tengah lamunan itu, aku tersentak ketika mendengar bunyi decit beberapa kursi ditarik di dekatku. Tampaknya kawan-kawan yang kutunggu-tunggu sedari tadi secara tidak sengaja bertemu di tengah jalan, kemudian menghampiriku berbarengan. Aku meletakkan ponselku dan berdiri untuk menyambut. Salah seorang dari mereka berempat---diselimuti rasa ingin tahu berlebih---agaknya ingin membuatku tersinggung dengan mengintip beranda ponselku yang didapati sebuah ikon obrolan daring, ``Oh, rupanya sudah jera dengan kehidupan \textit{hikikomori}-mu itu?'' lalu duduk meringis di salah satu kursi tidak jauh dari tempatku. Semuanya telah mengisi posisinya masing-masing seperti sediakala; menyisakan sebuah kursi yang seharusnya ditempati seorang sahabat kami yang paling berisik.

``Bagaimana kabarmu, Egg? Ah, tentu saja baik-baik saja,'' jawab Lee seraya tersenyum lebar, akan tetapi air mukanya adalah perlainan. Dia tidak sedang berusaha menyembunyikan gelagat khawatirnya. Sebagai mantan perwira, barang tentu jari-jarinya kuasa menahan rasa pedih dan pilu. Pun awaknya yang perkasa. Namun roman mukanya, itu hal lain.

``Aku tidak apa-apa, wahai kawanku yang baik. Lihatlah, aku tidak kekurangan sesuatu pun!'' seraya membuka lenganku lebar-lebar.

Lee mengangguk kecil menghormatiku. Kali ini lekuk kecil pada pipinya berbicara tulus; menunjukkan gigi-gigi putihnya yang dirawat sangat baik. Di antara kami semua, aku dulu benar-benar tampak teramat nelangsa. Mereka mengerti betul tentang aku yang paling kehilangan di sini.

``Pastilah baik-baik saja. Lihatlah apa yang telah diperbuatnya dengan kentang-kentang gorengku!'' tukas Ara dengan sikap canggung menggelitik yang dibuat-buatnya itu. Terkadang seorang alan-alan tidak sanggup menghadirkan pertunjukkan dengan lurus.

``Hei, itu memang milikku!'' seruku menimpali.

Dengan cekatan dia mulai menarik beberapa potong kentang goreng. ``Sekarang, ini jadi milikku.''

``Sialan kau ini!''

Kami pun tertawa terpingkal-pingkal. Aku secara kebetulan melupakan nyaris semua yang diperbincangkan setelah itu. Yang jelas, kami banyak-banyak tertawa hingga sakit perut. Lalu sampai pada topik yang tidak disangka-sangka.

``Wah, kamu mau menikah? Serius?''

``Ah, canggung kalau begitu. Kami pikir kamu tidak akan laku hingga akhirnya memohon-mohon padaku.'' Ara tergelak-gelak.

``Dih, siapa juga yang mau dengan badut sepertimu.'' Suara tawa Asse pun berderai menyusul.

``Ceritakan dong, Asse. Siapa gerangan orangnya? Apakah kelak kamu pindah juga?''

``Hmmm... Kupikir tunanganku, Bert, itu orang dari bagian timur. Tapi kami akan pergi ke utara nantinya. Kami tentunya tidak akan menunggu jadwal vaksinasi menurut kabar yang tidak jelas usutannya. Kudengar ia beroleh pekerjaan yang lebih bagus di sana. Kalau tidak salah, tempatnya di perbatasan. Ah, tidak perlu risau akan hal itu. Meskipun kerap terdengar warta yang tidak mengenakkan dari area sana, tapi aku pernah pergi ke sana sekali dan menemui mereka itu ramah-ramah semua. Hmmm... Yah, begitu kurasa,'' sembari memutar bola matanya ke atas.

``Kok seperti tidak yakin begitu?''

``Yah, bagaimana pun kalau mengenai Bert,'' dia menghela napas, ``aku tidak tahu percis seperti apa orangnya. Dia tidak pernah membicarakan jati dirinya, apalagi keluarganya. Bukankah bergaul dengannya akan mengadakan peristiwa-peristiwa mendebarkan?'' Asse cekikikan sendiri.

Kami mengiranya ia cuma bercanda. Barangkali semua penggila misteri akan seperti itu.

``Ih, serius kok!''

Ternyata dia benar-benar tidak masuk akal. Kami pun hanya menggeleng-gelengkan kepala. Dalam suasana seperti itu, aku berniat mencandai Karin yang biasanya tidak banyak bicara, ``Jadi kamu kapan, Karin?'' Dia tersenyum sipu lantas menjawab, ``Tak usah terburu-buru. Katakan saja kapan kamu siapnya,'' dengan agak jenaka.

``Hah?''

Sontak kami semua menganga; mempertontonkan paras yang sama sekali tidak mengerti dengan keajaiban ini. Tetiba Karin memberikan sebuah kode. Seorang pelayan paruh baya muncul tiba-tiba dari arah belakangku membawakan sepiring ubi rebus dengan roman tak senang. Lee berkata tanpa gentar, ``Tiada seorang pun dari kami memesannya, pak.'' Pelayan itu tertegun dan memicingkan mata pada Karin---yang telah sigap menurunkan topi bisbolnya hingga menutupi separuh wajahnya---lalu pergi mengambil jalan memutar.

Setelah pelayan itu tak tampak lagi batang hidungnya di balik sudut ruangan, Karin mencondongkan badannya ke arah kami dan mulai berbisik-bisik.

``Dia dulu dengan ayahku di perbatasan.''

Kami serempak terkejut. Rupanya anak-beranaknya pun dari perbatasan.

``Dia sangat keras terhadapku sewaktuku kecil. Ayah bilang, dia selalu mencemaskanku dengan menanyakan hal-hal sepele. Kiranya hari ini, dia masih belum sudi memaafkan kesembronoan ayahku sebagai penyintas dari pasukan garis depan. Semenjak itu, dia pensiun dini sebagai prajurit terhormat dan bekerja di sebuah perusahaan jawatan yang entah di mana demi mengirimiku biaya penghidupan dan sekolah sampai tamat. Terkejut bukan main melihatnya di sini sekarang. Dialah yang terbaik.''

``Aha, begitu. Dia amat khawatir jika melepaskan nona belia kepada seseorang---yang mentereng jauh dari kata---semacam Egg.''

Ara tertawa terbahak-bahak, disusul Karin dan teman-teman yang lain; wajahku memerah padam. Selepas itu kami memutuskan untuk membubarkan reuni kecil ini dengan baik-baik. Belum tentu akan ada lagi yang semisal ini dalam satu dasawarsa ke depan. Bagaimanapun, aku berbahagia hari ini; tiada akan lagi ihwal larat hati yang terbetik mendobrak masuk menutuh buku harian kami. Akan kutandaskan segalanya dengan benar dan teratur, aku berjanji. Dengan segala senang hati kusalami mereka satu per satu. Kecuali Lee; dia pulang bersamaku.

Kami memang satu arah hingga persimpangan kedua nanti. Cukup ganjil menatap Lee---yang sedianya ceria---kali ini tidak berbicara sepatah kata pun; sesaat aku merasa tak nyaman. Meskipun dia tersenyum padaku, tidak sedikit pun terbesit padanya untuk membuka mulut duluan; diam seribu basa. Aku sama sekali tak mampu menerka isi kepalanya. Kami lanjut membungkam tak berkutik, mengayunkan kaki bersama-sama setahap demi setahap bak arak-arakan yang pulang sehabis kehilangan penontonnya. ``Apakah seorang bekas opsir memang begitu adanya? Inikah rasanya mengunyah segelas es batu di bawah panas terik matahari?'' tanyaku sendiri, ``Kalau bangkai galikan kuburnya, kalau hidup sediakan buaiannya.'' Sebelum kami berpisah, dia mendekapku sembari mengeluarkan amplop coklat kusam berukuran sedang berbungkus kantong plastik bening yang dilipat-lipat dari saku kiri mantelnya, kemudian menyelipkannya ke dalam saku rompi hudiku secepat kilat sementara tangan satunya menggenggam erat bahuku yang ringkih. Seusai menyerapahi pelaku yang mencekik Isha terlalu rapi, dia memperingatiku, ``Bermain air basah, bermain api letup, bermain pisau luka.'' Aku semakin terheran-heran dibuatnya. Setengah sadar, aku membalas, ``Matahari itu bolehkah ditutup dengan nyiru?'' Dia berkelit kecut; aku membalik pulang dengan penuh kemenangan.

Sesudah terlampau jauh dari pengawasan Lee, aku menikung ke sebuah peternakan yang tertinggal habis akibat pajak taksah dari segerombolan jahanam di distrik terpencil itu. Cukup jelas ingatanku di sana tatkala seseorang tua renta membopong sebuah pasu meloncati sekat bambu sambil berteriak tanpa menoleh, ``Terkutuklah ibumu!'' kepada sejumlah orang yang mengitari pekarangan tanpa berupaya memburunya lebih jauh. Aku mematung nyaris tak bernapas karena berang. Sejak hari itu, orang tua itu hilang tak pernah kembali. Pun tidak ada surat kabar yang  mempercakapkannya. Dia lenyap bagai ditelan bumi. Aku sungguh-sungguh merindukan tawa lepasnya saat menyambut hangat kepulangan kami bersekolah.

Orang tua itu memang gemar sekali bermain kata. Jikalau sudah masuk waktunya makan siang, tak dapat tidak Isha dan teman-teman sebayanya dibuatnya canggung dengan kuis-kuis yang sukar dijawab oleh anak-anak perempuan berusia sepuluh tahun. Aku hanya duduk tersungkur menahan geli melihat bibir merahnya mencibir. Tapi lama-lama, hilang geli karena gelitik. Isha pun berhasil melontarkan cangkriman balik; tetua itu berdecak kagum dalam kekalahannya. Maka dibuangnya kelakarnya itu tak bersisa.

Di lain waktu, kami berdua pernah memergoki orang tua itu sedang merutuk seraya menggali-gali tanah menggunakan pencedok berkarat. Aku dan Isha saling memandang kebingungan; tidak ada yang berani memulai tegur sapa. Sekonyong-konyong anak perempuan yatim piatu di sebelahku terisak-isak pilu. Menyadari keberadaan anak angkuknya berdiri berjauhan, dia meminta maaf dengan suara paraunya, ``Ah, maafkan daku, Isha. Orang tua ini sama sekali tidak bermaksud membuatmu berdukacita,'' sambil menundukkan pandangan. Diambilnya sebuah kotak besi di bawah kakinya lalu dikuburnya dalam-dalam. Orang sepuh itu mendekat, berlutut dan memeluk Isha dengan sesal bercampur haru. Kemudian ia menengadah kepadaku, ``Barang bila mendapati sesuatu terjadi padaku, kau ambillah kotak tadi,'' sambil berkaca-kaca memohon. Aku mengangguk tanda mengerti.

Kupikir setelah belasan tahun, barang itu akan tetap abadi di sana. Ternyata benar, masih utuh di dalam kotak pandora. Yakni sebuah buku dengan alamat yang ditulis dengan huruf besar-besar; mencuat darinya potongan-potongan surat kabar tentang seorang anak yang mati ditebas para bajingan bermulut besar. Ngeri. Lalu kupanjatkan syukur berbilang kali sambil bergegas mengantongi buku itu kemudian lanjut pulang.

Belum genap empat puluh kaki, aroma hujan telah memenuhi rongga penciumku. Lalu disusul rintik-rintik hujan yang jatuh mengenai atap-atap rumah berlapis seng, tertabuh secara bergantian. Mengalunkan irama khas dari pinggiran kota yang kian lama kian tak menentu dan akhirnya menjadi gemuruh. Derap langkahku menggema di lorong-lorong samar lagi becek. Sekejap kemudian, bunyi desing peluru berkaliber 5,6 mm menyeruak di antara semak belukar, diikuti deru mesin tua yang dipaksa berlari tergopoh-gopoh. Sama sekali tiada pekik-pekuk terdengar, pun nyanyian burung camar di balik hujan deras. Aku berlari dan terus berlari.

\separator{}

\noindent Aku mendengkus melihat kesemuanya tergeletak di atas meja tulisku yang agak lapuk, lalu mengekeh. Sebab cerita rakyat dan dongeng anak-anak tidak semestinya menggunakan kata ganti orang pertama. Seluruh rentetan kejadian yang begitu tidak bisa dipercaya ini membuat kering mulutku, lalu merayap ke kerongkongan. Semakin jauh kupikirkan, semakin tak masuk akal. Meninggalkan mulas yang tak tertahankan. Kedua mataku perih; jantungku meledak-ledak. ``Aku butuh menenangkan diri.'' Lantas pergi mencuci muka dan mengambil dua kaleng minuman bersoda dari lemari pendingin.

Dua jam berlalu, hujan mulai mereda, tak akan ada pelangi. Juga tanda-tanda laporan berita akan dibacakan di mana pun yang mengulas mengenai kejadian sore tadi, penembakan di titik buta. Kala itu memang aku baru mau mencapai mulut gang; bergerak dari jalur yang redup dan sesak. Jadi yang bisa kulakukan hanyalah meraba-raba dinding sekitar, kali saja mau tersandung oleh potongan akar pohon yang menyembul dari bawah kaki. Segalanya begitu sayup, gelap dan kemudian lelap.

% Lee memberiku lembaran-lembaran kosong dari Isha yang ditulis dengan tangan kirinya menggunakan huruf stenografi yang kehilangan tanda bacanya

% Aku membacanya keras-keras aksara sendu yang diantarkan pertapa gila, ``Semboyan yang muluk-muluk itu tidak banyak gunanya, lebih baik bertindak.''

Aku berusaha tetap terjaga. Semasih belum aku tertidur di atas sofa lipat, meluruskan tulang-tulang punggung, mengemuli diri dengan selimut tebal, mencari-cari posisi ternyaman untuk mengasokan diri dari pegal-pegal, aku mulai mengecek kembali gadget kesayangan sebentar. Setelah acuh tak acuh menggulirkan beranda Twitter-ku berkali-kali dengan usapan jempol, akhirnya aku terpikat oleh sebuah tulisan dari seorang teman yang tidak dikenal. Sebuah kisah yang belum tentu benar, namun membuatku amat khawatir.

% \chapter{Ponsel Lipat}

% Hari itu, 14 Desember menurut almanak para penyembah berhala, aku berjalan terkantuk-kantuk sekembalinya dari laut lepas. Malam sebelumnya, kudapati diriku tidak bisa tertidur lantaran satu-satunya layar bubutan terjebak di dalam terpaan badai yang menderam. Tonggaknya roboh terhempas tepat mengenai hulu kemudi...

% begin back matter

\end{document}
% END THE DOCUMENT
