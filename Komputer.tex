% universal settings
\documentclass[smalldemyvopaper,11pt,twoside,onecolumn,openright,extrafontsizes]{memoir}
\usepackage[utf8x]{inputenc}
\usepackage[T1]{fontenc}
\usepackage[osf]{Alegreya,AlegreyaSans}

% PACKAGE DEFINITION
% typographical packages
\usepackage{microtype} % for micro-typographical adjustments
\usepackage{setspace} % for line spacing
\usepackage{lettrine} % for drop caps and awesome chapter beginnings
\usepackage{titlesec} % for manipulation of chapter titles

% for placeholder text
\usepackage{lipsum} % to generate Lorem Ipsum

% other
\usepackage{calc}
\usepackage{hologo}
\usepackage[hidelinks]{hyperref}
%\usepackage{showframe}
\usepackage{soul}

% PHYSICAL DOCUMENT SETUP
% media settings
\setstocksize{8.5in}{5.675in}
\settrimmedsize{8.5in}{5.5in}{*}
\setbinding{0.175in}
\setlrmarginsandblock{0.611in}{1.222in}{*}
\setulmarginsandblock{0.722in}{1.545in}{*}

% defining the title and the author
%\title{\LaTeX{} ePub Template}
%\title{\textsc{How I Started to Love {\fontfamily{cmr}\selectfont\LaTeX{}}}}
\title{Komputer}
\author{Naru Aika}
\newcommand{\ISBN}{0-000-00000-2}
\newcommand{\press}{}

% custom second title page
\makeatletter
\newcommand*\halftitlepage{\begingroup % Misericords, T&H p 153
  \setlength\drop{0.1\textheight}
  \begin{center}
  \vspace*{\drop}
  \rule{\textwidth}{0in}\par
  {\Large\textsc\thetitle\par}
  \rule{\textwidth}{0in}\par
  \vfill
  \end{center}
\endgroup}
\makeatother

% custom title page
\thispagestyle{empty}
\makeatletter
\newlength\drop{}
\newcommand*\titleM{\begingroup % Misericords, T&H p 153
  \setlength\drop{0.15\textheight}
  \begin{center}
  \vspace*{\drop}
  \rule{\textwidth}{0in}\par
  {\HUGE\textsc\thetitle\par}
  \rule{\textwidth}{0in}\par
  {\Large\textit\theauthor\par}
  \vfill
  {\Large\scshape\press}
  \end{center}
\endgroup}
\makeatother

% chapter title manipulation
% padding with zero
\renewcommand*\thechapter{\ifnum\value{chapter}<10 0\fi\arabic{chapter}}
% chapter title display
\titleformat
{\chapter}
[display]
{\normalfont\scshape\huge}
{\HUGE\thechapter\centering}
{0pt}
{\vspace{18pt}\centering}[\vspace{42pt}]

% typographical settings for the body text
\setlength{\parskip}{0em}
\linespread{1.09}

% HEADER AND FOOTER MANIPULATION
  % for normal pages
  \nouppercaseheads{}
  \headsep = 0.16in
  \makepagestyle{mystyle}
  \setlength{\headwidth}{\dimexpr\textwidth+\marginparsep+\marginparwidth\relax}
  \makerunningwidth{mystyle}{\headwidth}
  \makeevenhead{mystyle}{}{\textsf{\scriptsize\scshape\thetitle}}{}
  \makeoddhead{mystyle}{}{\textsf{\scriptsize\scshape\leftmark}}{}
  \makeevenfoot{mystyle}{}{\textsf{\scriptsize\thepage}}{}
  \makeoddfoot{mystyle}{}{\textsf{\scriptsize\thepage}}{}
  \makeatletter
  \makepsmarks{mystyle}{%
  \createmark{chapter}{left}{nonumber}{\@chapapp\ }{.\ }}
  \makeatother
  % for pages where chapters begin
  \makepagestyle{plain}
  \makerunningwidth{plain}{\headwidth}
  \makeevenfoot{plain}{}{}{}
  \makeoddfoot{plain}{}{}{}
  \pagestyle{mystyle}
% END HEADER AND FOOTER MANIPULATION

% table of contents customisation
\renewcommand\contentsname{\normalfont\scshape Daftar Isi}
\renewcommand\cftchapterfont{\normalfont}
\renewcommand{\cftchapterpagefont}{\normalfont}
\renewcommand{\printtoctitle}{\centering\Huge}

% layout check and fix
\checkandfixthelayout{}
% \fixpdflayout

% custom
\newcommand\separator{
  \begin{center}
    \(\ast~\ast~\ast\)
  \end{center}
}

% BEGIN THE DOCUMENT
\begin{document}
\pagestyle{empty}
% the half title page
% \halftitlepage
% \cleardoublepage
% the title page
\titleM{}
\clearpage
% copyright page
% \noindent{\small{This novel is entirely a work of fiction. The names, characters and incidents portrayed in it are the product of the author's imagination. Any resemblance to actual persons, living or dead, or events or localities is entirely coincidental.\par\vfill\noindent Paperback Edition\space\today\\ISBN\space\ISBN\\\copyright\space\theauthor. All rights reserved.\par\vfill\noindent\theauthor\space asserts the moral right to be identified as the author of this work. All rights reserved in all media. No part of this publication may be reproduced, stored in a retrieval system, or transmitted, in any form, or by any means, electronic, mechanical, photocopying, recording or otherwise, without the prior written permission of the author and/or the publisher.\par}}
% \clearpage

% dedication
% \begin{center}
% \itshape{\noindent{Untuk para penyuka novel picisan.}}
% \end{center}

% begin front matter
\frontmatter{}
\pagestyle{mystyle}
% preface
% \chapter*{Kata Pengantar}
% \lipsum[100-104]
% acknowledgements
% \chapter*{Ucapan Terima Kasih}
% \lipsum[1-9]
% table of contents
% \clearpage
% \tableofcontents*

% begin main matter
\mainmatter{}

\chapter*{Prolog}

Malam tak berangin itu, aku terbaring kosong bagai onggokan batang padi yang mengering di atas rerumputan tak bertuan di tepi jalan setapak yang berlubang; sebentar-sebentar memandang jauh ke langit yang terbentang luas pada hari kesatu. Terbenam ke dalam kesusahan hati yang berliku-liku tak menentu dari kumpulan kata tak berkesudahan yang menghunjam kencang ke dalam relung hatiku. ``Daripada mendengarkan banyak omong kosong,'' pikirku, ``lebih baik kulihat saja sendiri berkeliling sambil memohon secarik uraian terang yang cukup nyaring untuk menampik sangkaan tak segan manusia tentang diriku yang konon telah kecanduan mengonsumsi gadget.'' Namun tanpa disadari di kala itu aku justru masih menaruh tatapan erat ke dalam kelir-kelir menganjur dari layar gadget dan mengakhiri kalimat ini sambil menggerutu. ``Bah, betapa menyedihkannya aku ini,'' gumamku, ``Baiklah, kalau begitu, sebaliknya saja, mengapa orang-orang---'' (namun aku kurang senang untuk mempertanyakan diriku sendiri, sehingga tersebutlah \textit{orang-orang} menggantikan \textit{daku}) ``---sampai ketagihan, ya?'' Dalam benakku muncul sejumlah alasan dari ujaran-ujaran para tetua yang tak sampai lagi terdengar oleh muda mudi dalam satu dekade terakhir ini.

\hyphenation{pe-ne-ri-ma}

``Misalnya saja,'' tulisku, ``dengan ponsel terhubung ke jejaring, orang-orang bisa bertukar pesan dengan siapa pun di seluruh penjuru dunia tanpa perlu memikirkan \textit{hari bulan kapan sampainya}, karena akan secara instan tanpa adanya hari bervakansi. Bukan pula \textit{berapa koin ongkos kirimnya}, sebab akan sangat murah sekalipun lintas negara. Atau \textit{akan kepada siapa sampainya}, karena tidak akan terselip atau salah kirim kecuali salah alamat.'' Lalu aku menambahkan, ``Namun, di pihak yang sama, mereka lekas mengeluhkan hal-hal sepele. Misalnya, \textit{balasannya kapan datangnya}, padahal belum lama dibaca oleh si penerima. Atau \textit{berapa orang dikirimnya}, sebab sering kali itu hanya sebatas pesan siar acak. Atau \textit{dengan siapa pengirimnya}, karena terkadang itu adalah pesan anonim. Kemudian mulailah bermunculan ide-ide aplikasi olah pesan iseng semacam Slowly yang kemarin kuinstal.''

\hyphenation{me-nyim-pan me-nyi-pit-kan be-lum mu-ngil-nya te-ri-ma ber-ki-sah}

Tiba-tiba ponselku berdering ketibaan malam kian melarut, memekik di tengah kesunyian riang-riang yang sudah pulas. Dalam sekejap tercabut dari penguasaan lengahku untuk beroleh sebuah arloji tak bermerek yang terbungkus rapi di dalam kotaknya. Aku lupa bila sedang ditemani seseorang hampir sebaya saat itu. Dia menyipitkan matanya yang jelita sembari tersenyum memiringkan kepalanya lalu berkata, ``Selamat ulang tahun, Egg! Dua puluh tahun ini sungguh menyenangkan!'' biarpun ia belum setua itu. Entah apalah kehendaknya, seakan-akan kami berdua sama-sama memangku memori episodik dalam rentang waktu sedemikian panjangnya. Dengan keanggunannya tak bercela, dia mengajukan tangan mungilnya ke arahku, ``Ayo?'' Namun kuurungkan maksud untuk meraihnya seraya merunduk menyimpan gugup riuh nadiku dan menolaknya lembut, ``Tidak perlu. Terima kasih, ya.'' Dia pun menariknya kembali, mengembungkan pipi merahnya, merajuk tanpa kata, lantas bergegas menjauhiku. Enggan-enggan, aku bangkit menyiuk tidak berdaya lalu berjalan merapah---sembari memperhatikan lekat-lekat punggungnya dekat-dekat---menuju ke sekumpulan makhluk ijtimaiah yang sedang duduk-duduk melingkari api unggun sementara keasyikan berkisah kasih tentang kehidupan sekolahan dulu.

Berulang tahun memang tidak pernah menyenangkan, semuanya mengacungkan tangannya tanda setuju. Betapa tidak, kebanyakan mereka yang berumah di sekitarku lekas sekali mengeluarkan cek demi memestakan hiruk pikuk sisa hidupnya. Padahal boleh jadi besok atau lusa, orang yang diberinya selamat padam terkulai di atas usungan. Oleh karena itu, perkumpulan ini tidak dimaksudkan untuk merayakan sesuatu pun, meskipun kelihatannya masih bukan begitu. ``Biarlah saja, toh tidak ada untungnya juga bagiku mencampuri urusan semacam ini,'' pikirku. Sepatutnya aku lebih peduli terhadap gerak tawanya yang merdu dalam kesukaan perjamuan kecil-kecilan ini. Rasanya tidak pernah aku berbuat sedikit pun memikirkan perihal kehilangan hal itu. ``Jangan sampai, jangan sampai,'' ulangku. Lalu hilang membisu dalam bingkaian seroja.

\chapter{Gula Batu}

\hyphenation{ber-tong-kat akhir-nya}

Di suatu pagi nan cerah---seusai hujan tak berjeda mengguyur di antara lenyapnya dua matahari di ufuk barat; membanjiri seperdua pemukiman di kota ini yang tidak memiliki drainase yang mumpuni---aku berhimpun kembali bersama teman-teman di sebuah kafetaria untuk sebuah janji yang tertunda. Sebagai seorang aku, jelas kutelah datang paling dahulu dengan setelan kasual yang kusut tak sempat kusterika. Sembari duduk bertongkat lutut selama enggang menggeram di sebuah pojokan, menikmati sepiring kentang goreng yang baru saja dimasak, sesekali menghirup secangkir kopi hitam nan pahit, sekilas aku teringat sebuah peristiwa lima belas tahun silam saat seorang sahabat paling karib telah memaksa kami membuat rencana untuk datang kembali ke sini segera sesudah vaksinasi diumumkan. Beberapa pekan setelahnya, aku berbisik di dekat sahabatku yang rupawan tutur katanya itu, ``Seseorang yang membuat janji tidak seharusnya mati lebih dulu, kan?'' Dia hanya diam tak berkutik. Pada akhirnya janji hanya sebatas janji; kami biarkan ia melanggar sendiri ucapan kesediaan yang telah terpatri sekian lama. Pandemi berkepanjangan ini sudah membuat sedu sedan rakyat jelata nestapa dalam perkabungan; menciutkan tekad mereka untuk tidak pergi keluar rumah. Masa depan tidak mungkin diketahui. Namun untuk diprakirakan, itu barangkali bisa terjadi jika sebab musababnya tidak saru dan bisa ditelaah. Misalnya, \textit{berapa butir} apel yang dimakan utuh dengan bijinya untuk dapat membunuh seseorang. Atau \textit{berapa lama} mayat akan menjadi kejur lalu melunak lagi.

\hyphenation{me-la-ku-kan pe-me-rin-tah-an}

Dengar-dengar, seperangkat komputer mampu melakukan semua perhitungan ramalan secepat kilat. Umpamanya badan klimatologi---yang rutin diperdengarkan melalui siaran radio, televisi, dan laman resmi pemerintahan sipil---yang mampu memprediksi kapan ladang-ladang kami akan mendapati hujan dengan menganalisis pola kecepatan dan arah angin musiman, derajat kelengasan udara, temperatur titik embun, dan sebagainya. Bahkan kalau mau spesifik, terdapat ungkapan-ungkapan semisal \textit{hujan lebat} ketika curah hujan mencapai lebih dari 40 milimeter per jam, \textit{hujan badai} ketika disertai dengan angin yang melaju sekian kilometer per jam, dan \textit{hujan rintik-rintik}.

\hyphenation{ke-rin-du-an-ku se-be-ra-pa}

Namun bagaimana dengan \textit{seberapa indah} hari-hariku nanti tanpa seorang pun dari sanak kerabatku. Atau \textit{seberapa nyata} kerinduanku terhadap kawan-kawanku? Sememangnya, perkara abstrak seperti itu tidak memiliki satuan ukur. Yang mungkin ada hanyalah rasa yang terukir gamblang dari lagak lagam seseorang. Namun kiranya tetap bisa diukur juga, yakni dengan menghitung seberapa lama kami melepas rindu, misalnya. Taruhlah jika kali ini kami menetap selama empat puluh menit, sebagaimana yang biasa kami lakukan pada perjamuan rutin sekali sepekan, maka tidak ada bedanya. Jika lewat dari itu, berarti ialah rindu menahun.

Terlambat kusadari, semangkuk kecil bungkalan gula batu---di samping secangkir minuman yang nyaris tak bersisa---itu terus-menerus menatapku. Di era yang serba canggih ini, aku kembali termenung, akankah nantinya tempat ini dapat mengingat kesukaan setiap pelanggannya dalam menakar gula, lalu menerapkannya ke dalam kudapan yang dipesan? Itu mungkin saja akan membuat bersuka hati anak-anak perempuan yang melanggan tiap Sabtu. Kuingin cepat-cepat mengambil potret bungkalan itu dengan kamera saku lalu mengirimkannya ke alat cetak jauh di kamar bacaku untuk kujadikan referensi tulisanku berikutnya, tetapi nahas terjatuh entah di mana.

\hyphenation{men-de-ngar}

Di tengah lamunan itu, aku tersentak ketika mendengar bunyi decit beberapa kursi ditarik di dekatku. Tampaknya kawan-kawan yang kutunggu-tunggu sedari tadi secara tidak sengaja bertemu di tengah jalan, kemudian menghampiriku berbarengan. Aku meletakkan ponselku dan berdiri untuk menyambut. Salah seorang dari mereka berempat---diselimuti rasa ingin tahu berlebih---agaknya ingin membuatku tersinggung dengan mengintip beranda ponselku yang didapati sebuah ikon obrolan daring, ``Oh, rupanya sudah jera dengan kehidupan \textit{hikikomori}-mu itu?'' lalu duduk meringis di salah satu kursi tidak jauh dari tempatku. Semuanya telah mengisi posisinya masing-masing seperti sediakala; menyisakan sebuah kursi yang seharusnya ditempati seorang sahabat kami yang paling berisik.

``Bagaimana kabarmu, Egg? Ah, tentu saja baik-baik saja,'' jawab Lee seraya tersenyum lebar, akan tetapi kerut dahinya adalah perlainan. Dia tidak sedang berusaha menyembunyikan gelagat khawatirnya. Sebagai mantan perwira, barang tentu jari-jarinya kuasa menahan rasa pedih dan pilu. Pun awaknya yang perkasa. Namun tidak bagi hatinya yang lembut.

``Aku tidak apa-apa, wahai kawanku yang baik. Lihatlah, aku tidak kekurangan sesuatu pun!'' seraya membuka lenganku lebar-lebar.

Lee mengangguk kecil menghormatiku. Kali ini lekuk kecil pada pipinya berbicara tulus; menunjukkan gigi-gigi putihnya yang dirawat sangat baik. Di antara kami semua, aku dulu benar-benar tampak teramat nelangsa. Mereka mengerti betul tentang aku yang paling kehilangan di sini atau begitulah yang kukira.

``Pastilah baik-baik saja. Lihatlah apa yang telah diperbuatnya dengan kentang-kentang gorengku!'' tukas Ara dengan sikap canggung menggelitik yang dibuat-buatnya itu. Terkadang seorang alan-alan tidak sanggup menghadirkan pertunjukkan dengan lurus.

``Hei, itu memang milikku!'' seruku menimpali.

Dengan cekatan dia mulai menarik beberapa potong kentang goreng. ``Sekarang, ini jadi milikku.''

``Sialan kau ini!''

Kami pun tertawa terpingkal-pingkal. Aku secara kebetulan melupakan nyaris semua yang diperbincangkan setelah itu. Yang jelas, kami banyak-banyak tertawa hingga sakit perut. Lalu sampai pada topik yang tidak disangka-sangka.

``Wah, kamu mau menikah? Serius?''

``Ah, syukurlah kalau begitu. Kami pikir kamu tidak akan laku hingga akhirnya memohon-mohon padaku.'' Ara tergelak-gelak.

``Dih, siapa juga yang mau dengan badut sepertimu.'' Suara tawa Asse pun berderai menyusul.

\hyphenation{a-pa-kah}

``Ceritakan dong, Asse. Siapa gerangan orangnya? Apakah kelak kamu pindah juga?''

\hyphenation{me-ne-mu-kan}

``\textit{Hmmm} \dots Kupikir tunanganku, Bert, itu orang dari timur. Tapi kami akan pergi ke utara nantinya. Kami tentunya tidak akan menunggu jadwal vaksinasi menurut kabar yang tidak jelas usutannya. Kudengar ia beroleh pekerjaan yang lebih bagus di sana. Kalau tidak salah, tempatnya di perbatasan. Ah, tidak perlu risau akan hal itu. Meskipun kerap terdengar warta yang tidak mengenakkan, tapi aku pernah mengunjunginya sekali dan menemukan bahwa mereka itu ramah-ramah semua. Yah, begitulah kurasa,'' sembari memutar bola matanya ke atas.

``Kok, seperti tidak yakin begitu?''

\hyphenation{meng-he-la}

``Yah, bagaimana pun kalau mengenai Bert,'' dia menghela napas, ``aku tidak tahu percis seperti apa orangnya. Dia tidak pernah membicarakan jati dirinya, apalagi keluarganya. Bukankah bergaul dengannya akan mengadakan peristiwa-peristiwa mendebarkan?'' Asse cekikikan sendiri.

Kami mengiranya ia cuma bercanda. Barangkali semua penggila misteri akan seperti itu.

``Ih, serius kok!''

\hyphenation{ter-se-nyum}

Ternyata dia benar-benar tidak masuk akal. Kami pun hanya menggeleng-gelengkan kepala. Dalam suasana seperti itu, aku berniat mencandai Karin yang biasanya tidak banyak bicara, ``Jadi kamu kapan, Karin?'' Dia tersenyum sipu lantas menjawab, ``Tak usah terburu-buru. Katakan saja kapan kamu siapnya,'' dengan agak jenaka.

``Hah?''

\hyphenation{se-nang me-mi-cing-kan}

Tetiba seorang pelayan muncul dari arah belakangku membawakan sepiring ubi rebus dengan roman tidak senang. Lee berkata tanpa gentar, ``Tiada seorang pun dari kami memesannya, Pak.'' Pelayan itu tertegun dan memicingkan mata pada Karin---yang telah sigap menurunkan topi bisbolnya hingga menutupi separuh wajahnya---lalu pergi mengambil jalan memutar.

Setelah pelayan itu tak tampak lagi batang hidungnya di balik sudut ruangan, Karin mencondongkan badannya ke arah kami dan mulai berbisik-bisik.

``Dia dulu dengan ayahku di perbatasan.''

\hyphenation{ti-dak}

Kami serempak terkejut sekaligus prihatin. Rupanya anak-beranaknya telah banyak menelan garam. Setiap dari kami pun mengetahui bahwa eksistensi penduduk di kawasan perbatasan mempengaruhi penilaian terhadap pelaksanaan kedaulatan suatu negara. Kecemburuan terhadap pembangunan sosial-ekonomi yang tidak berimbang dengan negara tetangga serta dinamika hubungan yang tidak simetris dengan pemerintah pusat menjadi penyebab tidak jarangnya persengketaan keluarga berujung perpindahan kewarganegaraan secara permanen. Tidak ayal lagi bila Karin selama ini menjadi pemurung.

``Dia sangat keras terhadapku sewaktuku kecil. Ayah bilang, dia selalu mencemaskanku dengan menanyakan hal-hal sepele. Kiranya hari ini, dia masih belum sudi memaafkan kesembronoan ayahku sebagai penyintas dari pasukan garis depan. Semenjak itu, dia pensiun dini sebagai prajurit terhormat dan bekerja di sebuah perusahaan jawatan yang entah di mana demi mengirimiku biaya penghidupan dan sekolah sampai tamat. Terkejut bukan main melihatnya di sini sekarang. Dialah yang terbaik.''

``Aha, begitu. Jadi beliau amat khawatir melepaskan nona belia ini kepada seseorang---yang mentereng jauh dari kata---semacam Egg.''

Ara tertawa terbahak-bahak, disusul Karin dan teman-teman yang lain; wajahku bersemu merah. Selepas itu, kami memutuskan untuk membubarkan reuni kecil ini dengan baik-baik. Belum tentu akan ada lagi yang semisal ini dalam satu dasawarsa ke depan. Bagaimanapun, aku berbahagia hari ini; tiada akan lagi ihwal larat hati yang terbetik mendobrak masuk menutuh buku harian kami. Akan kutandaskan segalanya dengan benar dan teratur, aku berjanji. Dengan segala senang hati kusalami mereka satu per satu. Kecuali Lee; dia pulang bersamaku.

\hyphenation{men-de-kap-ku me-nye-lip-kan-nya}

Kami memang satu arah hingga persimpangan kedua nanti. Cukup ganjil menatap Lee---yang sedianya ceria---kali ini tidak berbicara sepatah kata pun; sesaat aku merasa tak nyaman. Meskipun dia sesekali tersenyum padaku, tidak sedikit pun terbesit padanya untuk membuka mulut duluan; diam seribu basa. Aku sama sekali tak mampu menerka isi kepalanya. Kami lanjut saling bungkam tak berkutik, mengayunkan kaki bersama-sama setahap demi setahap bak arak-arakan yang pulang sehabis kehilangan penontonnya. ``Apakah seorang bekas opsir memang begitu adanya? Inikah rasanya mengunyah segelas es batu di bawah panas terik matahari?'' tanyaku sendiri, ``Kalau bangkai galikan kuburnya, kalau hidup sediakan buaiannya.'' Sebelum kami berpisah, dia mendekapku sembari mengeluarkan amplop cokelat kusam berukuran sedang---yang dilipat-lipat di dalam bungkus plastik bening---dari saku kiri mantelnya, kemudian menyelipkannya ke dalam saku rompi hudiku secepat kilat sementara tangan satunya menggenggam erat bahuku yang ringkih. Seusai menyerapahi pelaku yang mencekik Isha terlalu rapi, dia memperingatiku, ``Bermain air basah, bermain api letup, bermain pisau luka.'' Aku semakin terheran-heran dibuatnya. Setengah sadar, aku membalas, ``Matahari itu bolehkah ditutup dengan nyiru?'' Dia berkelit kecut; aku membalik pulang dengan penuh kemenangan.

\hyphenation{pe-ka-ra-ngan}

Sesudah terlampau jauh dari pengawasan Lee, aku menikung ke sebuah peternakan yang tertinggal habis akibat pajak taksah dari segerombolan jahanam di distrik terpencil itu. Cukup jelas ingatanku di sana tatkala seseorang tua renta membopong sebuah pasu meloncati sekat bambu sambil berteriak tanpa menoleh, ``Terkutuklah ibumu!'' kepada sejumlah orang yang mengitari pekarangan tanpa berupaya memburunya lebih jauh. Aku mematung nyaris tak bernapas karena berang. Kejadian itu tepat dua tahun setelah Isha berpulang. Sejak hari itu, orang tua itu hilang tak pernah kembali. Pun tidak ada surat kabar yang  mempercakapkannya. Lenyap bagai ditelan bumi. Aku sungguh-sungguh merindukan tawa lepasnya saat menyambut hangat kepulangan kami bersekolah.

Orang tua itu memang gemar sekali bermain kata. Jikalau sudah masuk waktunya makan siang, tak dapat tidak Isha dibuatnya canggung dengan kuis-kuis yang sukar dijawab oleh anak-anak perempuan berusia sepuluh tahun. Aku hanya duduk tersungkur menahan geli melihat bibir merahnya mencibir. Tapi lama-lama, habis geli oleh gelitik. Pada suatu waktu, Isha pun berhasil melontarkan cangkriman balik; tetua itu berdecak kagum dalam kekalahannya. Maka dibuangnya kelakarnya itu tak bersisa.

Di lain waktu, kami berdua pernah memergoki orang tua itu sedang merutuk seraya menggali-gali tanah menggunakan pencedok berkarat. Aku dan Isha saling memandang kebingungan; tidak ada yang berani memulai tegur sapa. Sekonyong-konyong anak perempuan yatim piatu di sebelahku terisak-isak pilu. Menyadari keberadaan anak angkuknya berdiri berjauhan, dia meminta maaf dengan suara paraunya, ``Ah, maafkan daku, Isha. Orang tua ini sama sekali tidak bermaksud membuatmu berdukacita,'' sambil menundukkan pandangan. Diambilnya sebuah kotak besi di bawah kakinya lalu dikuburnya dalam-dalam. Orang sepuh itu mendekat, berlutut dan memeluk Isha dengan sesal bercampur haru. Kemudian ia menengadah kepadaku, ``Barang bila mendapati sesuatu terjadi padaku, kau ambillah kotak tadi,'' sambil berkaca-kaca memohon. Aku mengangguk tanda mengerti.

Kupikir setelah belasan tahun, barang itu akan tetap abadi di sana. Ternyata benar, masih utuh di dalam kotak pandora. Yakni sebuah buku dengan alamat yang ditulis dengan huruf besar-besar; mencuat darinya potongan-potongan surat kabar tentang seorang anak yang mati ditebas para bajingan bermulut besar. Ngeri. Lalu kupanjatkan doa berbilang kali sambil bergegas mengantongi buku itu kemudian lanjut pulang.

Belum genap empat puluh kaki, aroma hujan telah memenuhi rongga penciumku. Lalu disusul rintik-rintik hujan yang jatuh mengenai atap-atap rumah berlapis seng, tertabuh secara bergantian. Mengalunkan irama khas dari pinggiran kota yang kian lama kian tak menentu dan akhirnya menjadi gemuruh. Derap langkahku menggema di lorong-lorong samar lagi becek. Sekejap kemudian, bunyi desing peluru berkaliber 5,6 mm menyeruak di antara semak belukar, diikuti deru mesin tua yang dipaksa berlari tergopoh-gopoh. Sama sekali tiada pekik-pekuk terdengar, pun nyanyian burung camar di balik hujan deras. Aku berlari dan terus berlari menuju mulut gang; bergerak dari jalur yang redup dan sesak. Sambil meraba-raba dinding sekitar, kali saja mau tersandung oleh potongan akar pohon yang menyembul dari bawah kaki. Segalanya begitu sayup, gelap dan kemudian lelap.

\separator{}

\hyphenation{ke-rong-ko-ngan}

\noindent Seluruh rentetan kejadian yang nyaris tidak bisa dipercaya ini membuat kering mulutku, lalu merayap ke kerongkongan. Semakin jauh kupikirkan, semakin tak masuk akal. Meninggalkan mulas yang tak tertahankan. Kedua mataku perih; jantungku meledak-ledak. ``Tenanglah,'' titahku sendiri. Lantas pergi mencuci muka dan mengambil dua kaleng minuman bersoda dari lemari pendingin.

\hyphenation{ma-sih}

Kupandangi arloji yang selalu kubawa-bawa ke mana pun aku pergi itu berlama-lama. Jarum penunjuknya masih saja berputar terbalik. Dua jam berlalu, hujan mulai mereda, tak akan ada pelangi. Juga tiada tanda-tanda laporan berita akan dibacakan di mana pun yang mengulas mengenai kejadian sore tadi, penembakan di titik buta. Aku menghela napas. Dua jam berlalu kembali.

\hyphenation{ku-te-ri-ma}

Di bawah temaram bulan yang mengintip di sela-sela gorden kamar apartemenku, aku mendengkus melihat kesemuanya tergeletak di atas meja tulisku, lalu mengekeh tanpa bermaksud pongah atau apa. Dokumen yang kuterima dari Lee sangat menarik perhatianku. Tidak lain tidak bukan ialah sebuah draf buku cerita anak-anak yang pada saat itu tengah ditulis oleh Isha.

Isha Soraya. Jelas bukan nama seorang putri pribumi. Ibunya berkebangsaan Bangladesh, sedangkan leluhur ayahnya berasal dari keluarga buruh India yang diikutsertakan dalam migrasi besar pada pertengahan abad ke-19. Hingga usia sembilan tahun, Isha tinggal di Lubok Antu. Pada tahun berikutnya, kedua orang tua Isha dipindahtugaskan dalam rangka diplomasi pengaturan kerja sama pembangunan antara negara-negara anggota D-8. Sejak saat itu, Isha diasuh oleh kerabat angkatnya di Putussibau, di mana kami pertama kali bertemu tepat ketika memasuki tahun ajaran baru.

% Rujukan:
% - https://id.wikipedia.org/wiki/India_Malaysia
% - https://mitra.gov.my/sejarah-masyarakat-india-malaysia/

\hyphenation{ber-ha-rap}

Bertahun-tahun tinggal bersama \textit{amang}-nya, Isha mulai terbawa-bawa senang dengan literatur. Jangankan berharap memiliki perpustakaan, sekolah kami tidak mampu menyediakan buku-buku bacaan apa pun selain buku paket dari pemerintah. Buku-buku dipinjaminya dari sahabat \textit{amang}-nya yang notabene wartawan. Tiap kali aku, Asse, Lee, Ara, dan Karin menyamperinya untuk main, kuperhatikan Isha senantiasa duduk-duduk di bawah naungan pohon kersen favoritnya. Masih mengenakan seragam sekolahnya, dia lekas hanyut dalam bacaannya sendiri.

``Nanti dulu \dots''

``Sebentar lagi \dots''

Terkadang \textit{nanti} dan \textit{sebentar}-nya itu menjadi \textit{tidak pernah}. Begitulah yang kerap kali terjadi. Katanya, untuk menjadi penulis yang hebat, kita harus rajin-rajin membaca. Oleh karena itu, aku menduga bahwa Isha akan memilih bahasa dan sastra, tetapi ia malah mengambil jurusan sosial dan politik.

Pada malam kematiannya, editornya mengirimkannya kembali melalui faksimile untuk diperbaiki. Kecuali di bagian-bagian kecil dari setiap bab, coretan di sana sini memenuhi satu halaman terakhir dari epilognya sehingga setiap komentarnya bahkan ceritanya sendiri nyaris tidak terbaca. % Seolah-olah si editor telah berputus asa terhadap dirinya yang gagal meyakinkan Isha untuk tidak menuliskan cerita penutup semacam itu.

Dari yang pernah kudengar, intuisi editor profesional itu luar biasa. Mereka mampu mengetahui psikologi sang penulis dengan cukup akurat dari tulisan-tulisannya. Sebab para penulis yang baru memulai debut mereka kerap kali hanya akan menuliskan hal-hal yang dirasanya benar saja. Namun aku tidak mengira bahwa hal tersebut berlaku dalam kasus Isha kali ini. Sama sekali tidak ada komentar mengenai gramatikanya; segalanya tertuju pada diksinya, yang mana kritikan semacam ini tidak lazim dilontarkan oleh editor kepada penulis yang berpengalaman. Karena jika tidak hati-hati, itu bisa saja mengubah gaya penulisan yang sepatutnya unik.

\hyphenation{mung-kin-kah}

Ada semacam pola dari yang kuyakini mengalir di dalam tulisan-tulisan tangan sang editor yang berpadu dengan huruf-huruf cetak si pengarang naskah. Seakan-akan keduanya sedang beradu mulut satu sama lain. Mungkinkah sebuah dialog tersemat di balik kata-kata yang tidak selaras ini? Barang tentu ini tidak akan mudah, sebab Lee telah diminta untuk memecahkan teka-teki draf kopian ini sebagai detektif konsultan polisi. Sayangnya, dia tidak menyertakan pendapatnya dalam amplop yang kuterima ini.

\hyphenation{ku-ma-suk-kan}

Mengabaikan huruf-huruf stenografi sang editor yang tidak keruan ujungnya, halaman tersebut lekas kumasukkan ke dalam mesin pindai untuk kemudian kupisahkan bagian isinya dari komentarnya dengan program editor grafis raster. Melalui serangkaian proses pengenalan karakter optik, gambar teks tikan tersebut berhasil kukonversi menjadi teks yang dapat dikenali, diedit, dan dicari oleh komputer. Lalu aku mulai membacakan penggalan awal kisah penutup itu seperti yang nenekku biasa lakukan ketika menidurkanku di pangkuannya sewaktuku kecil.

% Kata-kata yang sia-sia, tidak selaras, atau kata-kata hilang

\textit{Kedua daun jendela kubuka lebar-lebar, agar cahaya mentari dapat masuk menyinari sudut-sudut ruang kamar bacaku. Aku mengambil beberapa roti apit yang kuisikan telur ceplok, selada bokor, dan keju mozarela dari dapur, lalu menyusunnya ke dalam satu keranjang piknik hingga penuh. Tidak lupa, sepatu bot, topi bundar, dan syal musim semi kukenakan. Lalu aku berjalan tegap keluar sambil bersenandung riang. ``Selamat pagi, Lili!'' sapa seekor anjing gembala. ``Selamat pagi, Tom!''}

\hyphenation{la-yak-nya ke-pa-lang}

Aku---yang seringnya membolos di kelas bahasa layaknya murid-murid nakal lainnya---kebingungan bukan kepalang mencari di mana letak kejanggalan dari penggalan kisah tersebut. Aku sama sekali tidak mampu; tidak tanpa catatan sang editor. Dengan mencocokkan bagian isi cerita dengan coretan-coretannya yang telah kurapikan sedikit, aku dapat melihat dengan lebih jelas pola-pola yang kuyakini benar adanya. Contohnya, cerita rakyat dan dongeng pengantar tidur tidak semestinya menggunakan sudut pandang orang pertama. Setidaknya, begitulah yang selalu kutemukan.

% \chapter{Daging Asap}

% Aku berusaha tetap terjaga. Semasih belum aku tertidur di atas sofa lipat, meluruskan tulang-tulang punggung, mengemuli diri dengan selimut tebal, mencari-cari posisi ternyaman untuk mengasokan diri dari otot-otot yang pegal serta pikiran yang buncah, aku mulai mengecek kembali gadget kesayangan sebentar. Setelah acuh tak acuh menggulirkan beranda Twitter-ku berkali-kali dengan usapan jempol, akhirnya aku terpikat oleh sebuah tulisan dari seorang teman yang tidak dikenal. Sebuah kisah yang belum tentu benar, namun membuatku amat khawatir. Sayangnya, kantuk telanjur datang menemuiku.

% Hari itu, 14 Desember menurut almanak para penyembah berhala, aku berjalan dengan masih terkantuk-kantuk sekembalinya dari laut lepas. Malam sebelumnya, kudapati diriku tidak bisa tertidur lantaran satu-satunya layar bubutan terjebak di dalam terpaan badai yang menderam. Tonggaknya roboh terhempas tepat mengenai hulu kemudi. ``Ah, mengerikan benar,'' ucapku tak sadar.

% Kuserahkan beberapa koin login untuk sebuah koran pagi yang lembaran-lembarannya masih hangat itu berlama-lama. Aku masih saja tidak bisa membayangkan maksud dari tajuk utamanya, tetapi aku menangkap resumenya, yaitu zaman telah berkembang lebih cepat daripada nalar orang lamban ini.

% Juga selembar koran harian yang mengabari tentang sebuah riset pemerintah dalam mengurangi tingkat kematian penduduk kota dengan mengembangkan sebuah sistem yang mampu memprediksi kapan seseorang akan bunuh diri.

% \chapter{Mangga Kuweni}

% Egg mengirimiku sebuah buku kosong bertajuk \textit{Seseorang Berangkat Menuju Surga} dengan guntingan-guntingan pers yang kutemukan terselip di dalamnya. Entah mengapa, aku merasa familiar dengan tulisan tangannya ...

% Aku membacanya keras-keras aksara sendu yang diselipkan di bawah daun pintu kamarku oleh si pertapa gila, ``Semboyan yang muluk-muluk itu tidak banyak gunanya, lebih baik bertindak.''

% \chapter{Roti Mentega}

% Kami tahu, ini akan menjadi pukulan berat bagi Karin \dots

% ``Semua kita tahu bahwa Isha baik kepada semua orang, tetapi \dots''

% ``Aku tahu betul, cokelat ini akan menjadi sebab terbunuhnya diriku. Namun kumakan langsung tanpa ragu ... untuk Ara.''

% ``Kecuali dirimu telah duduk di sampingnya, Isha tidak pernah tertawa lepas di hadapan teman-temannya, sekalipun bukan dikau yang sedang berkelakar.''

% begin back matter

\end{document}
% END THE DOCUMENT
