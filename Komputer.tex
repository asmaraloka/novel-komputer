% universal settings
\documentclass[smalldemyvopaper,11pt,twoside,onecolumn,openright,extrafontsizes]{memoir}
\usepackage[utf8x]{inputenc}
\usepackage[T1]{fontenc}
\usepackage[osf]{Alegreya,AlegreyaSans}

% PACKAGE DEFINITION
% typographical packages
\usepackage{microtype} % for micro-typographical adjustments
\usepackage{setspace} % for line spacing
\usepackage{lettrine} % for drop caps and awesome chapter beginnings
\usepackage{titlesec} % for manipulation of chapter titles

% for placeholder text
\usepackage{lipsum} % to generate Lorem Ipsum

% other
\usepackage{calc}
\usepackage{hologo}
\usepackage[hidelinks]{hyperref}
%\usepackage{showframe}
\usepackage{soul}

% PHYSICAL DOCUMENT SETUP
% media settings
\setstocksize{8.5in}{5.675in}
\settrimmedsize{8.5in}{5.5in}{*}
\setbinding{0.175in}
\setlrmarginsandblock{0.611in}{1.222in}{*}
\setulmarginsandblock{0.722in}{1.545in}{*}

% defining the title and the author
%\title{\LaTeX{} ePub Template}
%\title{\textsc{How I Started to Love {\fontfamily{cmr}\selectfont\LaTeX{}}}}
\title{Komputer}
\author{Naru Aika}
\newcommand{\ISBN}{0-000-00000-2}
\newcommand{\press}{}

% custom second title page
\makeatletter
\newcommand*\halftitlepage{\begingroup % Misericords, T&H p 153
  \setlength\drop{0.1\textheight}
  \begin{center}
  \vspace*{\drop}
  \rule{\textwidth}{0in}\par
  {\Large\textsc\thetitle\par}
  \rule{\textwidth}{0in}\par
  \vfill
  \end{center}
\endgroup}
\makeatother

% custom title page
\thispagestyle{empty}
\makeatletter
\newlength\drop{}
\newcommand*\titleM{\begingroup % Misericords, T&H p 153
  \setlength\drop{0.15\textheight}
  \begin{center}
  \vspace*{\drop}
  \rule{\textwidth}{0in}\par
  {\HUGE\textsc\thetitle\par}
  \rule{\textwidth}{0in}\par
  {\Large\textit\theauthor\par}
  \vfill
  {\Large\scshape\press}
  \end{center}
\endgroup}
\makeatother

% chapter title manipulation
% padding with zero
\renewcommand*\thechapter{\ifnum\value{chapter}<10 0\fi\arabic{chapter}}
% chapter title display
\titleformat
{\chapter}
[display]
{\normalfont\scshape\huge}
{\HUGE\thechapter\centering}
{0pt}
{\vspace{18pt}\centering}[\vspace{42pt}]

% typographical settings for the body text
\setlength{\parskip}{0em}
\linespread{1.09}

% HEADER AND FOOTER MANIPULATION
  % for normal pages
  \nouppercaseheads{}
  \headsep = 0.16in
  \makepagestyle{mystyle}
  \setlength{\headwidth}{\dimexpr\textwidth+\marginparsep+\marginparwidth\relax}
  \makerunningwidth{mystyle}{\headwidth}
  \makeevenhead{mystyle}{}{\textsf{\scriptsize\scshape\thetitle}}{}
  \makeoddhead{mystyle}{}{\textsf{\scriptsize\scshape\leftmark}}{}
  \makeevenfoot{mystyle}{}{\textsf{\scriptsize\thepage}}{}
  \makeoddfoot{mystyle}{}{\textsf{\scriptsize\thepage}}{}
  \makeatletter
  \makepsmarks{mystyle}{%
  \createmark{chapter}{left}{nonumber}{\@chapapp\ }{.\ }}
  \makeatother
  % for pages where chapters begin
  \makepagestyle{plain}
  \makerunningwidth{plain}{\headwidth}
  \makeevenfoot{plain}{}{}{}
  \makeoddfoot{plain}{}{}{}
  \pagestyle{mystyle}
% END HEADER AND FOOTER MANIPULATION

% table of contents customisation
\renewcommand\contentsname{\normalfont\scshape Daftar Isi}
\renewcommand\cftchapterfont{\normalfont}
\renewcommand{\cftchapterpagefont}{\normalfont}
\renewcommand{\printtoctitle}{\centering\Huge}

% layout check and fix
\checkandfixthelayout{}
% \fixpdflayout

% custom
\newcommand\separator{
  \begin{center}
    \(\ast~\ast~\ast\)
  \end{center}
}

% BEGIN THE DOCUMENT
\begin{document}
\pagestyle{empty}
% the half title page
% \halftitlepage
% \cleardoublepage
% the title page
\titleM{}
\clearpage
% copyright page
% \noindent{\small{This novel is entirely a work of fiction. The names, characters and incidents portrayed in it are the product of the author's imagination. Any resemblance to actual persons, living or dead, or events or localities is entirely coincidental.\par\vfill\noindent Paperback Edition\space\today\\ISBN\space\ISBN\\\copyright\space\theauthor. All rights reserved.\par\vfill\noindent\theauthor\space asserts the moral right to be identified as the author of this work. All rights reserved in all media. No part of this publication may be reproduced, stored in a retrieval system, or transmitted, in any form, or by any means, electronic, mechanical, photocopying, recording or otherwise, without the prior written permission of the author and/or the publisher.\par}}
% \clearpage

% dedication
% \begin{center}
% \itshape{\noindent{Untuk para penyuka novel picisan.}}
% \end{center}

% begin front matter
\frontmatter{}
\pagestyle{mystyle}
% preface
% \chapter*{Kata Pengantar}
% \lipsum[100-104]
% acknowledgements
% \chapter*{Ucapan Terima Kasih}
% \lipsum[1-9]
% table of contents
% \clearpage
% \tableofcontents*

% begin main matter
\mainmatter{}

\chapter*{Prolog}

\hyphenation{me-nge-ring la-ngit me-ngan-jur mem-per-ta-nya-kan}

Pada suatu malam tak berangin yang panjang, aku terbaring kosong bagai onggokan batang padi yang mengering di atas rerumputan jepang di tepi jalan setapak yang berlubang. Sebentar-sebentar memandang jauh ke langit yang terbentang luas pada hari kesatu. Hening, sepi, dan gelap; tiada suara binatang dan kerlip bintang-bintang. Sedari pukul sebelas, sebagaimana pula yang lazim terjadi baru-baru ini, aku telah terbenam ke dalam kesusahan hati yang berliku-liku tak menentu dari kumpulan kata tak berkesudahan yang terhunjam kencang di dalam relung hatiku.

\hyphenation{me-ngon-sum-si me-ne-mu-kan}

Aku telah mencoba berkeliling ke sana ke mari demi secarik uraian terang guna menampik sangkaan tak segan manusia tentang diriku yang konon telah kecanduan mengonsumsi gadget. Namun tanpa disadari, di kala itu aku justru masih menaruh tatapan erat ke dalam kelir-kelir menganjur dari layar ponsel dan mengakhiri kalimat ini sambil menggerutu. ``Bah, betapa menyedihkannya aku ini. Baiklah,'' ucapku mengalah, ``jika tidak berhasil menemukan pembenaran, sebaliknya saja, mengapa orang-orang---'' (aku tidak senang untuk mempertanyakan diriku sendiri, sehingga tersebutlah \textit{orang-orang} menggantikan \textit{daku}) ``---sampai ketagihan, ya?'' Dalam benakku muncul sejumlah alasan dari ujaran-ujaran para tetua yang tak sampai lagi terdengar oleh muda mudi dalam satu atau dua dekade terakhir ini.

\hyphenation{me-re-ka}

Selepas menahan tombol undur spasi hingga segenap bakal utasku lenyap, aku kembali sibuk memilih-milih diksi. ``Dengan ponsel terhubung ke jejaring,'' tulisku cepat tanpa suara memakai papan tik multisentuh, ``setiap kita bisa bertukar pesan dengan siapa pun di seluruh penjuru dunia. Cepat, murah, dan sedikit kekhawatiran. Tiada lagi si pengantar yang absen gara-gara cuti bersama atau vakansi akhir pekan, amplop yang terselip atau hilang di tengah jalan, dan perangko yang lepas atau rusak dalam kepergian lintas benua.''

\hyphenation{me-nge-luh-kan ku-ins-tal}

Namun di pihak yang sama, mereka mulai mengeluhkan hal-hal lain. Mulai dari singkatan-singkatan ajaib, ambigu, dan \textit{lebay}; pesan anonim \textit{mama minta pulsa}; \textit{kredit tanpa agunan} dan \textit{hadiah cek tunai}; pesan siar acak yang tidak menyebut-nyebut nama si penerima secara khusus; dan balasan yang tak kunjung datang, padahal baru saja sedetik lalu pesan itu dibuka. Maka demi menghindari salah sangka, atau sekadar melarikan diri, sebagian orang menonaktifkan tanda terima \textit{telah dibaca} dan \textit{terakhir terlihat}. Sebagian lainnya merasa kecewa dengan sensasi surat-menyurat kontemporer, maka terciptalah aplikasi olah pesan semisal Slowly yang kemarin baru kuinstal kendati belum sempat kujelajahi lebih dalam.

\hyphenation{ke-ti-ba-an}

Belum sempat menekan tombol pos yang terletak tepat di bawah kolom status, tiba-tiba ponselku berdering ketibaan malam kian melarut, memekik di tengah kesunyian riang-riang yang sudah pulas. Dalam sekejap, tercabut dari penguasaan lengahku untuk beroleh sebuah arloji tak bermerek meski terbungkus di dalam kotaknya. Aku lupa perempuan bersama sebuah sedang ditemani polos seseorang hampir sebaya saat itu. Dia menyipitkan matanya yang jelita, memiringkan kepalanya sembari tersenyum. ``Selamat ulang tahun, Egg! Dua puluh tahun ini sungguh menyenangkan!'' biarpun ia belum setua itu. Entah apalah kehendaknya, seakan-akan kami berdua sama-sama memangku memori episodik dalam rentang waktu sedemikian panjangnya.

\hyphenation{ber-ki-sah}

Dengan keanggunannya tak bercela, dia mengajukan tangan mungilnya ke arahku, ``Ayo?'' Namun kuurungkan maksud untuk meraihnya seraya merunduk menyimpan gugup riuh nadiku dan menolaknya lembut, ``Tidak perlu. Terima kasih, ya.'' Dia pun menariknya kembali, mengembungkan pipinya bak pauh dilayang, merajuk tanpa kata, lantas beranjak menjauhiku. Enggan-enggan aku bangkit, menyiuk tidak berdaya, lalu berjalan merapah---sembari memperhatikan lekat-lekat punggungnya dekat-dekat---menuju ke sekumpulan makhluk ijtimaiah yang sedang duduk-duduk melingkari api unggun sementara keasyikan berkisah kasih tentang kehidupan sekolahan dulu.

Berulang tahun memang tidak pernah menyenangkan, semuanya mengiakan. Betapa tidak, kebanyakan mereka yang berumah di sekitarku lekas sekali mengeluarkan cek demi memestakan hiruk pikuk sisa hidupnya. Terkadang beberapa botol arak jawa dibuka dan mereka bersulang-sulangan dengan wajah tanpa dosa. Padahal boleh jadi besok atau lusa, orang yang diberinya selamat padam terkulai di atas usungan. Oleh karena itu, perkumpulan ini tidak dimaksudkan untuk merayakan sesuatu pun, meskipun kelihatannya masih bukan begitu. Biarlah saja, toh tidak ada untungnya juga bagiku mencampuri urusan semacam ini, pikirku. Sepatutnya aku lebih peduli terhadap gerak tawanya yang merdu dalam kesukaan perjamuan kecil-kecilan ini. Rasanya tidak pernah sekali pun memikirkan bagaimana rasanya kehilangan kenangan itu. Jangan sampai, jangan sampai, ulangku. Lalu hilang membisu dalam bingkaian seroja.

\chapter{Bakpao Isi}

\hyphenation{ber-tong-kat}

Di suatu pagi nan cerah---seusai hujan tak berjeda mengguyur di antara lenyapnya dua matahari di ufuk barat; membanjiri seperdua pemukiman di kota ini yang tidak memiliki drainase yang mumpuni---aku berhimpun kembali bersama teman-teman di sebuah kafetaria untuk sebuah janji yang tertunda. Sebagai seorang aku, jelas kutelah datang paling dahulu dengan setelan kasual yang kusut tak sempat kusterika. Sembari duduk bertongkat lutut selama enggang menggeram di sebuah pojokan, menikmati sepiring kentang goreng yang baru saja dimasak, sesekali menghirup secangkir kopi hitam nan pahit, sekilas aku teringat sebuah peristiwa lima belas tahun silam saat seorang sahabat paling karib telah memaksa kami membuat rencana untuk datang kembali ke sini segera sesudah vaksinasi diumumkan.

\hyphenation{ti-dak nes-ta-pa}

Beberapa pekan setelahnya, aku berbisik di dekat sahabatku yang rupawan tutur katanya itu, ``Seseorang yang membuat janji tidak seharusnya mati lebih dulu `kan?'' Dia hanya diam tak berkutik. Pada akhirnya, janji hanya sebatas janji; kami biarkan ia melanggar sendiri ucapan kesediaan yang telah terpatri sekian lama. Pandemi berkepanjangan ini sudah membuat sedu sedan rakyat jelata nestapa dalam perkabungan; menciutkan tekad mereka untuk tidak pergi keluar rumah. Tidak ada seorang pun yang mengetahui masa depan dengan pasti. Namun barangkali itu bisa diprakirakan. Misalnya, \textit{berapa butir} biji apel yang dilumatkan ke dalam jem apel untuk sepotong roti panggang agar dapat membunuh seorang gadis kecil. Atau \textit{berapa lama} mayat yang diapungkan di laut mati akan menjadi kejur lalu melunak lagi. Ya, semua itu ada dalam buku-buku sains, kiranya. Namun, bagaimana pun juga, sahabat kami tidak mati dengan cara seperti itu.

\hyphenation{ber-be-da ber-pe-nga-la-man mu-sa-bab-nya}

Semua orang, termasuk yang sudah mati, gemar sekali berhitung nasib peruntungan di masa depan. Tentu berbeda dari sekadar mempercayai penderita eksem berpengalaman yang senantiasa tahu berapa menit lagi hujan akan turun di tempatnya berpijak, lantaran sebab musababnya saru dan tidak bisa ditelaah. Apalagi mempercayai orang pintar yang acap kali minta dibawakan segelas air putih untuk dijampi-jampi, mustahil \dots Kami pun tertawa terpingkal-pingkal memandangnya rendah.

\hyphenation{tem-pe-ra-tur}

Dewasa ini, seperangkat komputer mampu melakukan semua perhitungan ramalan secepat kilat. Umpamanya badan klimatologi dengan teknologi komputer miliknya mampu memprediksi kapan ladang-ladang kami akan mendapati hujan dengan menganalisis pola kecepatan dan arah angin musiman, derajat kelengasan udara, temperatur titik embun, dan sebagainya. Kalau mau spesifik, terdapat ungkapan-ungkapan semisal \textit{hujan lebat} ketika curah hujan mencapai lebih dari 40 milimeter per jam, \textit{hujan badai} ketika disertai dengan angin yang melaju sekian kilometer per jam, dan \textit{hujan rintik-rintik}. Sayangnya, kali ini yang datang bukanlah hujan rintik-rintik.

\hyphenation{ke-rin-du-an-ku se-be-ra-pa sa-tu-an}

\textit{Seberapa nyata} kerinduanku terhadap sahabat yang paling kucintai? \textit{Seberapa sulit} hari-hariku besok kujalani tanpanya? Sememangnya, perkara abstrak seperti itu tidak memiliki satuan ukur. Yang mungkin ada hanyalah rasa yang terukir gamblang dari lagak lagam seseorang. Namun kiranya tetap bisa diukur juga, yakni dengan menghitung seberapa lama kami melepas rindu, misalnya. Taruhlah jika kali ini kami menetap selama empat puluh menit, sebagaimana yang biasa kami lakukan pada perjamuan rutin sekali sepekan, maka tidak ada bedanya. Jika lewat dari itu, berarti ialah rindu menahun.

\hyphenation{me-ngi-rim-kan-nya}

Semangkuk kecil bungkalan gula batu---di samping secangkir minuman yang nyaris tak bersisa---itu terus-menerus menatapku dengan gemas, tatapan yang lurus macam bendul. Akankah nantinya tempat semacam ini dapat mengingat kesukaan setiap pelanggannya dalam menakar gula, lalu menerapkannya ke dalam kudapan yang dipesan pada kali-kali berikutnya? Itu mungkin saja akan membuat bersuka hati anak-anak perempuan yang melanggan tiap Sabtu. Kuingin cepat-cepat mengambil potret bungkalan itu dengan kamera saku lantas mengirimkannya ke alat cetak jauh di kamar bacaku, tetapi nahas terjatuh entah di mana. Kupikir Isha tetap akan menolak untuk menunggu hingga bosan berlarut-larut, karena pada akhirnya tetap saja kurang \dots ``Bukannya kurang, ini, sih, sama sekali tidak manis!'' tukasnya.

\hyphenation{men-de-ngar}

Di tengah lamunan itu, aku tersentak ketika mendengar bunyi decit beberapa kursi ditarik di dekatku. Tampaknya kawan-kawan yang kutunggu-tunggu sedari tadi secara tidak sengaja bertemu di tengah jalan, kemudian menghampiriku berbarengan. Aku meletakkan ponselku dan berdiri untuk menyambut. Salah seorang dari mereka berempat---diselimuti rasa ingin tahu berlebih---agaknya ingin membuatku tersinggung dengan mengintip beranda ponselku yang didapati sebuah ikon obrolan daring, ``Oh, rupanya sudah jera dengan kehidupan \textit{hikikomori}-mu itu?'' lalu duduk meringis di salah satu kursi tidak jauh dari tempatku. Semuanya telah mengisi posisinya masing-masing seperti sediakala; menyisakan sebuah kursi yang seharusnya ditempati seorang sahabat kami yang paling berisik.

``Bagaimana kabarmu, Egg? Ah, tentu saja baik-baik saja,'' jawab Lee seraya tersenyum lebar, akan tetapi kerut dahinya adalah perlainan. Dia tidak sedang berusaha menyembunyikan gelagat khawatirnya. Sebagai mantan perwira, barang tentu jari-jarinya kuasa menahan rasa pedih dan pilu. Pun awaknya yang perkasa. Namun tidak bagi hatinya yang selembut yoghurt.

``Aku tidak apa-apa, wahai kawanku yang baik. Lihatlah, aku tidak kekurangan sesuatu pun!'' seraya membuka lenganku lebar-lebar.

Lee mengangguk kecil menghormatiku. Kali ini lekuk kecil pada pipinya berbicara tulus; menunjukkan gigi-gigi putihnya yang dirawat sangat baik. Di antara kami semua, aku dulu benar-benar tampak teramat nelangsa. Mereka mengerti betul tentang aku yang paling kehilangan di sini, atau begitulah yang kukira.

``Pastilah baik-baik saja. Lihatlah apa yang telah diperbuatnya dengan kentang-kentang gorengku!'' tukas Ara dengan sikap canggung menggelitik yang dibuat-buatnya itu. Terkadang seorang alan-alan tidak sanggup menghadirkan pertunjukkan dengan lurus.

``Hei, itu memang milikku!'' seruku menimpali.

Dengan cekatan dia mulai menarik beberapa potong kentang goreng. ``Sekarang, ini jadi milikku.''

``Sialan kau ini!''

\hyphenation{su-a-sa-na-nya}

Kami pun tertawa terkial-kial. Aku secara kebetulan melupakan nyaris semua yang diperbincangkan setelah itu. Kendati dari dulu aku selalu yakin bahwa tidak semuanya perlu disimpan, namun aku sebenarnya ingin mengingat momen kali ini bersama mereka. Yang jelas, kami banyak-banyak tertawa hingga sakit perut, suasananya pun kian menghangat. Lalu sampai pada topik yang tidak disangka-sangka.

``Wah, kamu mau menikah? Serius?''

``Ah, syukurlah kalau begitu. Kami pikir kamu tidak akan laku hingga akhirnya memohon-mohon padaku.'' Ara tergelak-gelak.

``Dih, siapa juga yang mau dengan badut sepertimu.'' Suara tawa Asse pun berderai menyusul.

\hyphenation{a-pa-kah}

``Ceritakan dong, Asse. Siapa gerangan orangnya? Apakah kelak kamu pindah juga?''

\hyphenation{me-ne-mu-kan}

``\textit{Hmmm} \dots Kupikir tunanganku, Bert, itu orang dari timur. Tapi kami akan pergi ke utara nantinya. Kami tentunya tidak akan menunggu jadwal vaksinasi menurut kabar yang tidak jelas usutannya. Kudengar ia beroleh pekerjaan yang lebih bagus di sana. Kalau tidak salah, tempatnya di perbatasan. Ah, tidak perlu risau akan hal itu. Meskipun kerap terdengar warta yang tidak mengenakkan, tapi aku pernah mengunjunginya sekali dan menemukan bahwa mereka itu ramah-ramah semua. Yah, begitulah kurasa,'' sembari memutar bola matanya ke atas.

``Kok, seperti tidak yakin begitu?''

\hyphenation{meng-he-la}

``Yah, bagaimana pun kalau mengenai Bert,'' dia menghela napas, ``aku tidak tahu percis seperti apa orangnya. Dia tidak pernah membicarakan jati dirinya, apalagi keluarganya. Bukankah bergaul dengannya akan mengadakan peristiwa-peristiwa mendebarkan?'' Asse cekikikan sendiri.

Kami mengiranya ia cuma bercanda. Barangkali semua penggila misteri akan seperti itu.

``Ih, serius kok!''

\hyphenation{ter-se-nyum}

Ternyata dia benar-benar tidak masuk akal. Kami pun hanya menggeleng-gelengkan kepala. Dalam suasana seperti itu, aku berniat mencandai Karin yang biasanya tidak banyak bicara, ``Jadi kamu kapan, Karin?'' Dia tersenyum sipu lantas menjawab, ``Tak usah terburu-buru. Katakan saja kapan kamu siapnya,'' dengan agak jenaka.

``E-eh?'' Semuanya terlihat benar-benar terkejut kecuali Karin seorang diri.

\hyphenation{de-ngan}

Sesaat berikutnya, tetiba seorang pelayan muncul dari arah belakangku membawakan sepiring ubi rebus dengan roman tidak senang. Lee berkata tanpa gentar, ``Tiada seorang pun dari kami memesannya, Pak.'' Pelayan itu tertegun dan memicingkan mata pada Karin---yang telah sigap menurunkan topi bisbolnya hingga menutupi separuh wajahnya---lalu pergi mengambil jalan memutar.

Setelah pelayan itu tak tampak lagi batang hidungnya di balik sudut ruangan, Karin mencondongkan badannya ke arah kami dan mulai berbisik-bisik tanpa beban.

``Dia dulu dengan ayahku di perbatasan.'' Dia berhenti sejenak, lalu melanjutkan, ``Dia sangat keras terhadapku sewaktuku kecil. Ayah bilang, dia selalu mencemaskanku dengan menanyakan hal-hal sepele. Kiranya hari ini, dia masih belum sudi memaafkan kesembronoan ayahku sebagai penyintas dari pasukan garis depan. Semenjak itu, dia pensiun dini sebagai prajurit terhormat dan bekerja di sebuah perusahaan jawatan yang entah di mana demi mengirimiku biaya penghidupan dan sekolah sampai tamat. Terkejut bukan main melihatnya di sini sekarang. Dialah yang terbaik.''

\hyphenation{ti-dak}

Kami serempak terkejut sekaligus prihatin. Rupanya anak-beranaknya telah banyak menelan garam. Setiap dari kami pun mengetahui bahwa eksistensi penduduk di kawasan perbatasan mempengaruhi penilaian terhadap pelaksanaan kedaulatan suatu negara. Kecemburuan terhadap pembangunan sosial-ekonomi yang tidak berimbang dengan negara tetangga serta dinamika hubungan yang tidak simetris dengan pemerintah pusat menjadi penyebab tidak jarangnya persengketaan keluarga berujung perpindahan kewarganegaraan secara permanen. Tidak ayal lagi bila Karin selama ini menjadi pemurung.

Bagaimanapun yang terjadi, syukurlah, semua orang di sini sedang dalam suasana hati yang sangat baik, sebagaimana yang selalu Isha harapkan. Kecuali Karin, tampaknya dia telah berubah menjadi orang yang lebih terbuka. Andai saja Isha berkesempatan mengawasinya hari ini, dia tidak akan lagi mencemaskan sesuatu pun di dunia yang penuh dengan permainan ini.

``Aha, begitu. Jadi beliau amat khawatir melepaskan nona belia ini kepada seseorang---yang mentereng jauh dari kata---semacam Egg.''

Ara tertawa terbahak-bahak, disusul Karin dan teman-teman yang lain; wajahku bersemu merah. Selepas itu, kami memutuskan untuk membubarkan reuni kecil ini dengan baik-baik. Belum tentu akan ada lagi yang semisal ini dalam satu dasawarsa ke depan. Bagaimanapun, aku berbahagia hari ini; tiada akan lagi ihwal larat hati yang terbetik mendobrak masuk menutuh buku harian kami. Akan kutandaskan segalanya dengan benar dan teratur, aku berjanji. Dengan segala senang hati kusalami mereka satu per satu. Kecuali Lee; dia pulang bersamaku.

\hyphenation{me-ngu-nyah me-nge-lu-ar-kan sa-tu-nya bo-leh-kah}

Kami memang satu arah hingga persimpangan kedua nanti. Cukup ganjil menatap Lee---yang sedianya ceria---kali ini tidak berbicara sepatah kata pun; sesaat aku merasa tak nyaman. Meskipun dia sesekali tersenyum padaku, tidak sedikit pun terbesit padanya untuk membuka mulut duluan; diam seribu basa. Aku sama sekali tak mampu menerka isi kepalanya. Kami lanjut saling bungkam tak berkutik, mengayunkan kaki bersama-sama setahap demi setahap bak arak-arakan yang pulang sehabis kehilangan penontonnya. Apakah seorang bekas opsir memang begitu adanya? Begitu tidak sanggupnya dia menjalani hari-harinya tanpa ditemani anjing pelacak kesayangannya yang suka sekali diajaknya mengobrol? Tapi tak apa, seperti kata peribahasa, \textit{kalau bangkai galikan kuburnya, kalau hidup sediakan buaiannya.}

Sebelum kami berpisah, dia mendekapku sembari mengeluarkan amplop cokelat kusam berukuran sedang---yang dilipat-lipat di dalam bungkus plastik bening---dari saku kiri mantelnya, kemudian menyelipkannya ke dalam saku rompi hudiku secepat kilat sementara tangan satunya menggenggam erat bahuku yang ringkih. Seusai menyerapahi pelaku yang mencekik Isha terlalu rapi, dia memperingatiku, ``Bermain air basah, bermain api letup, bermain pisau luka.'' Aku semakin terheran-heran dibuatnya. Setengah sadar, aku membalas, ``Matahari itu bolehkah ditutup dengan nyiru?'' Dia berkelit kecut; aku membalik pulang dengan penuh kemenangan.

\hyphenation{pe-ka-ra-ngan}

Sesudah terlampau jauh dari pengawasan Lee, aku menikung ke sebuah peternakan yang tertinggal habis akibat pajak taksah dari segerombolan jahanam di distrik terpencil itu. Cukup jelas ingatanku di sana tatkala seseorang tua renta membopong sebuah pasu meloncati sekat bambu sambil berteriak tanpa menoleh, ``Terkutuklah ibumu!'' kepada sejumlah orang yang mengitari pekarangan tanpa berupaya memburunya lebih jauh. Aku mematung nyaris tak bernapas karena berang. Kejadian itu tepat dua tahun setelah Isha berpulang. Sejak hari itu, orang tua itu hilang tak pernah kembali. Pun tidak ada surat kabar yang  mempercakapkannya. Lenyap bagai ditelan bumi. Aku sungguh-sungguh merindukan tawa lepasnya saat menyambut hangat kepulangan kami bersekolah.

Orang tua itu memang gemar sekali bermain kata. Jikalau sudah masuk waktunya makan siang, tak dapat tidak Isha dibuatnya canggung dengan kuis-kuis yang sukar dijawab oleh anak-anak perempuan berusia sepuluh tahun. Aku hanya duduk tersungkur menahan geli melihat bibir merahnya mencibir. Tapi lama-lama, habis geli oleh gelitik. Pada suatu waktu, Isha pun berhasil melontarkan cangkriman balik; tetua itu berdecak kagum dalam kekalahannya. Maka dibuangnya kelakarnya itu tak bersisa.

\hyphenation{de-ngan}

Di lain waktu, kami berdua pernah memergoki orang tua itu sedang merutuk seraya menggali-gali tanah menggunakan pencedok berkarat. Aku dan Isha saling memandang kebingungan; tidak ada yang berani memulai tegur sapa. Sekonyong-konyong anak perempuan yatim piatu di sebelahku terisak-isak pilu. Menyadari keberadaan anak angkuknya berdiri berjauhan, dia meminta maaf dengan suara paraunya, ``Ah, maafkan daku, Isha. Orang tua ini sama sekali tidak bermaksud membuatmu berdukacita,'' sambil menundukkan pandangan. Diambilnya sebuah kotak besi di bawah kakinya lalu dikuburnya dalam-dalam. Orang sepuh itu mendekat, berlutut, memeluk Isha dengan sesal bercampur haru. Kemudian ia menengadah kepadaku, ``Barang bila mendapati sesuatu terjadi padaku, kau ambillah kotak tadi,'' sambil berkaca-kaca memohon. Aku mengangguk tanda mengerti.

Kupikir setelah belasan tahun, barang itu akan tetap abadi di sana. Ternyata benar, masih utuh di dalam kotak pandora. Yakni sebuah buku dengan alamat yang ditulis dengan huruf besar-besar; mencuat darinya potongan-potongan surat kabar tentang seorang anak yang mati ditebas para bajingan bermulut besar. Ngeri. Lalu kupanjatkan doa berbilang kali sambil bergegas mengantongi buku itu kemudian lanjut pulang.

\hyphenation{ber-ge-rak se-ga-la-nya po-to-ngan}

Belum genap empat puluh kaki, aroma hujan telah memenuhi rongga penciumku. Lalu disusul rintik-rintik hujan yang jatuh mengenai atap-atap rumah berlapis seng, tertabuh secara bergantian. Mengalunkan irama khas dari pinggiran kota yang kian lama kian tak menentu dan akhirnya menjadi gemuruh. Derap langkahku menggema di lorong-lorong samar lagi becek. Sekejap kemudian, bunyi desing peluru berkaliber 5,6 mm menyeruak di antara semak belukar, diikuti deru mesin tua yang dipaksa berlari tergopoh-gopoh. Sama sekali tiada pekik-pekuk terdengar, pun nyanyian burung camar di balik hujan deras. Aku berlari dan terus berlari menuju mulut gang; bergerak dari jalur yang redup dan sesak. Sambil meraba-raba dinding sekitar, kali saja mau tersandung oleh potongan akar pohon yang menyembul dari bawah kaki. Segalanya begitu sayup, gelap, dan kemudian lelap.

\separator{}

\hyphenation{ke-rong-ko-ngan}

\noindent Seluruh rentetan kejadian yang nyaris tidak bisa dipercaya ini membuat kering mulutku, lalu merayap ke kerongkongan. Semakin jauh kupikirkan, semakin tak masuk akal. Meninggalkan mulas yang tak tertahankan. Kedua mataku perih; jantungku meledak-ledak. Tenanglah, titahku sendiri. Lantas pergi mencuci muka dan mengambil dua kaleng minuman bersoda dari lemari pendingin.

\hyphenation{ma-sih}

Kupandangi arloji, yang selalu kubawa-bawa ke mana pun aku pergi, itu berlama-lama. Jarum penunjuknya masih saja berputar terbalik. Dua jam berlalu, hujan mulai mereda, tak akan ada pelangi. Juga tiada tanda-tanda laporan berita akan dibacakan di mana pun yang mengulas mengenai kejadian sore tadi: penembakan di titik buta. Aku menghela napas. Waktu berlalu kembali.

\hyphenation{ku-te-ri-ma}

Di bawah temaram bulan yang mengintip di sela-sela gorden kamar apartemenku, aku mendengkus melihat kesemuanya tergeletak di atas meja tulisku, lalu mengekeh tanpa bermaksud pongah atau apa. Dokumen yang kuterima dari Lee sangat menarik perhatianku. Tidak lain tidak bukan ialah sebuah draf buku cerita anak-anak yang pada saat itu tengah ditulis oleh Isha.

Isha Soraya. Jelas bukan nama seorang putri pribumi. Ibunya berkebangsaan Bangladesh, sedangkan leluhur ayahnya berasal dari keluarga buruh India yang diikutsertakan dalam migrasi besar pada pertengahan abad ke-19. Hingga usia sembilan tahun, Isha tinggal di Lubok Antu. Pada tahun berikutnya, kedua orang tua Isha dipindahtugaskan dalam rangka diplomasi pengaturan kerja sama pembangunan antara negara-negara anggota D-8. Sejak saat itu, Isha diasuh oleh kerabat angkatnya di Putussibau, di mana kami pertama kali bertemu tepat ketika memasuki tahun ajaran baru.

% Rujukan:
% - https://id.wikipedia.org/wiki/India_Malaysia
% - https://mitra.gov.my/sejarah-masyarakat-india-malaysia/

\hyphenation{ber-ha-rap}

Bertahun-tahun tinggal bersama \textit{amang}-nya, Isha mulai terbawa-bawa senang dengan literatur. Jangankan berharap memiliki perpustakaan, sekolah kami tidak mampu menyediakan buku-buku bacaan apa pun selain buku paket dari pemerintah. Buku-buku dipinjaminya dari sahabat \textit{amang}-nya yang notabene wartawan. Tiap kali aku, Asse, Lee, Ara, dan Karin menyamperinya untuk main, kuperhatikan Isha senantiasa duduk-duduk di bawah naungan pohon kersen favoritnya. Masih mengenakan seragam sekolahnya, dia lekas hanyut dalam bacaannya sendiri.

``Nanti dulu \dots''

``Iya, sebentar lagi \dots''

Terkadang \textit{nanti} dan \textit{sebentar}-nya itu menjadi \textit{tidak pernah}. Begitulah yang kerap kali terjadi. Katanya, untuk menjadi penulis yang hebat, kita harus rajin-rajin membaca. Oleh karena itu, aku menduga bahwa Isha akan memilih bahasa dan sastra, tetapi ia malah mengambil sarjana jurusan sosial dan politik.

Pada malam kematiannya, editornya mengirimkannya kembali melalui faksimile untuk diperbaiki. Kecuali di bagian-bagian kecil dari setiap bab, coretan di sana sini memenuhi satu halaman terakhir dari epilognya sehingga setiap komentarnya bahkan ceritanya sendiri nyaris tidak terbaca. % Seolah-olah si editor telah berputus asa terhadap dirinya yang gagal meyakinkan Isha untuk tidak menuliskan cerita penutup semacam itu.

Dari yang pernah kudengar, intuisi editor profesional itu luar biasa. Mereka mampu mengetahui psikologi sang penulis dengan cukup akurat dari tulisan-tulisannya. Sebab para penulis yang baru memulai debut mereka kerap kali hanya akan menuliskan hal-hal yang dirasanya benar saja. Namun aku tidak mengira bahwa hal tersebut berlaku dalam kasus Isha kali ini. Sama sekali tidak ada komentar mengenai gramatikanya; segalanya tertuju pada diksinya, yang mana kritikan semacam ini tidak lazim dilontarkan oleh editor kepada penulis yang berpengalaman. Karena jika tidak hati-hati, itu bisa saja mengubah gaya penulisan yang sepatutnya unik.

\hyphenation{mung-kin-kah}

Ada semacam pola dari yang kuyakini mengalir di dalam tulisan-tulisan tangan sang editor yang berpadu dengan huruf-huruf cetak si pengarang naskah. Seakan-akan keduanya sedang beradu mulut satu sama lain. Mungkinkah sebuah dialog tersemat di balik kata-kata yang tidak selaras ini? Barang tentu ini tidak akan mudah, sebab Lee telah diminta untuk memecahkan teka-teki draf kopian ini sebagai detektif konsultan polisi. Sayangnya, dia tidak menyertakan pendapatnya ke dalam amplop ini.

\hyphenation{ku-ma-suk-kan}

Mengabaikan huruf-huruf stenografi sang editor yang tidak keruan ujungnya, halaman tersebut lekas kumasukkan ke dalam mesin pindai untuk kemudian kupisahkan bagian isinya dari komentarnya menurut warna teksnya, hitam dari merah. Melalui serangkaian proses pengenalan karakter optik, gambar teks tikan tersebut berhasil kukonversi menjadi teks yang dapat dikenali, diedit, dan dicari oleh komputer. Sehingga komputerku dapat membacakanku penggalan awal kisah penutup itu seperti yang nenekku biasa lakukan ketika menidurkanku di pangkuannya sewaktuku kecil.

\textit{Kedua daun jendela kubuka lebar-lebar, agar cahaya mentari dapat masuk menyinari sudut-sudut ruang kamar bacaku. Aku mengambil beberapa roti apit yang kuisikan telur ceplok, selada bokor, dan keju mozarela dari dapur, lalu menyusunnya ke dalam satu keranjang piknik hingga penuh. Tidak lupa, sepatu bot, topi bundar, dan syal musim semi kukenakan. Lalu aku berjalan tegap keluar sambil bersenandung riang. ``Selamat pagi, Lili!'' sapa seekor anjing gembala. ``Selamat pagi, Tom!''}

\hyphenation{la-yak-nya ke-pa-lang}

Aku---yang seringnya membolos di kelas bahasa layaknya murid-murid nakal lainnya---kebingungan bukan kepalang mencari di mana letak kejanggalan dari penggalan kisah tersebut. Aku sama sekali tidak mampu; tidak tanpa catatan sang editor. Namun dengan mencocokkan bagian isi cerita dengan coretan-coretannya yang telah kurapikan sedikit, aku dapat melihat dengan lebih jelas pola-pola yang kuyakini benar adanya. Contohnya, cerita rakyat dan sebuah dongeng tidak semestinya menggunakan sudut pandang orang pertama. Setidaknya, begitulah yang selalu kami berdua sangka di bawah pohon kersen sambil menggenggam bakpao isi kacang merah.

\chapter{Bebek Kayu}

\hyphenation{ce-ri-ta}

Aku berusaha keras tetap terjaga. Semasih belum aku tertidur di atas sofa lipat yang sudah mulai agak kempis, meluruskan tulang-tulang punggung, mengemuli diri dengan selimut tebal, mencari-cari posisi ternyaman untuk mengasokan diri dari otot-otot yang pegal serta pikiran yang buncah, aku mulai mengecek kembali ponsel baruku sebentar. Setelah acuh tak acuh menggulirkan beranda Twitter-ku berkali-kali dengan usapan jempol, akhirnya aku terpikat oleh sebuah tulisan dari seorang teman yang tidak dikenal. Sebuah kisah yang belum tentu benar, namun membuatku amat khawatir. Sayangnya, kantuk telanjur datang menghampiriku, membiarkan ceritanya menggantung begitu saja tanpa sebuah pengakhiran.

\hyphenation{me-nge-na-i da-lam}

Hari itu, 14 Desember menurut almanak para penyembah berhala, terkantuk-kantuk aku menuruni geladak kapal sekembali dari laut lepas. Utas itu sudah tidak ada, juga ingatan dari sebagian ceritanya. Malam sebelumnya, kudapati diriku tidak bisa tidur sama sekali lantaran satu-satunya layar bubutan terjebak di dalam terpaan badai yang menderam. Tonggaknya roboh terhempas tepat mengenai hulu kemudi. ``Ah, mengerikan benar,'' ucapku tak sadar di depan seorang wanodya penjual koran keliling. Dia memandangiku dengan hormat layaknya melihat seorang ajudan. Kulitnya kecokelatan, khas orang pesisir. Topi kep berwarna merah muda tak lupa dikenakannya agar terlindung dari panas terik. Aku sedikit terpesona melihatnya tak berias, tetapi begitu manis memikat.

``Maaf?''

``Ah, bukan apa-apa. Saya hanya sedang mengoceh sendiri. Ha-ha-ha \dots''

Kuserahkan sejumlah koin logam dari kocek celanaku. Sebagai balasan, dia memberiku sebuah koran pagi yang masih hangat setiap lembarnya. Aku termangu sejenak menilik kover depannya, lalu kubolak-balikkan dengan tak sabar. Lalu mendesah kecil.

``Maaf?'' tanyanya kembali.

\hyphenation{sa-ngat}

``Oh, saya minta maaf. Kamu benar, ini masih amat sangat pagi untuk berkeluh kesah. Ha-ha-ha \dots''

\hyphenation{pe-ra-sa-an}

Kemudian aku beranjak meninggalkannya dengan perasaan campur aduk, atau dia yang terdahulu pergi, aku tidak lagi ingat. Sebentar-sebentar aku meliriknya kembali di balik surat kabar yang sedang kupegangi. Dia tidak putus-putus menawarkan dagangannya sambil berteriak, ``Koran \dots koran.'' Sementara aku masih saja berdiri mematung sebentar sebelum kemudian berusaha memaksakan diri memaknai tajuk utama harian itu sekilas.

\hyphenation{ter-khu-sus u-pa-ya}

Halaman surat kabar itu akhir-akhir ini terbagi atas enam lajur, padahal dahulunya ada sembilan lajur. Setiap alineanya pun ditulis pendek-pendek. Kiranya mereka telah menyadari bahwa masyarakat modern mulai meninggalkan bacaan-bacaan panjang. Meski begitu, terkhusus hari ini, sang kepala redaksi telah mewanti-wanti agar sebuah laporan yang agak panjang dimuat sebagai berita utamanya. Dua halaman penuh. Awalnya kupikir, dia terlalu terobsesi dengan yang satu ini. Ternyata, kabar itu memang menarik. Ia bercerita tentang salah satu upaya mutakhir pemerintah dalam mengurangi tingkat kematian penduduk kota. Yaitu dengan mengembangkan sebuah sistem terintegrasi yang kuasa memprediksi kapan seseorang akan bunuh diri. Benar-benar menarik. Namun, mula-mula aku perlu menemukan tempat singgah di sekitar pelabuhan sini yang paling cocok untuk memesan secangkir kopi lokal dan sepiring telur mata sapi. Ternyata mereka menawarkan daging asap juga, tidak buruk.

\hyphenation{ra-sa-nya}

Sehabis menyesap secangkir kopi tanpa gula dalam dua isapan, kubuka kembali lembaran surat kabar tadi. Sebentar kemudian mulai mengangguk-angguk. Dari apa yang dapat kutangkap sejauh ini, proyek kerja sama itu telah melibatkan banyak sekali pihak terkait, di antaranya adalah para pemangku jabatan pemerintahan, penyedia layanan komunikasi, ahli psikologi, pakar bahasa, dan ilmuwan komputer. Menurut kabar, tepat setelah satu hastawara, mereka akan memulai pengawasan ketat terhadap setiap tulisan yang beredar di media sosial. Hal yang demikian pastinya akan mengubrak-abrik keleluasaan pribadi setiap orang. Akan tetapi, siapa memangnya yang peduli dengan privasi? Ia tak ubahnya bak buah simalakama. Apakah seseorang lebih memilih untuk terlambat menghadiri pemakaman keluarganya atau membiarkan Facebook tahu makanan dan tempat favoritnya?

% Rujukan:
% - https://www.forbes.com/sites/neilsahota/2020/10/14/privacy-is-dead-and-most-people-really-dont-care/

\hyphenation{ja-ngan pe-run-dung-an}

Beberapa model kecerdasan artifisial, yang telah dilatih secara intensif menggunakan data relevan pengolahan bahasa alami, dipekerjakan dalam mengenali tulisan yang mengandung sentimen negatif. Kata-kata kasar dan tak senonoh, yang terucap setiap kali perundungan terjadi, mudah sekali dikenali. Daftar kata-kata itu tersedia di internet dalam semua bahasa, dapat diunduh secara cuma-cuma, sehingga para admin grup obrolan daring dapat mudah menyaringnya. Sedangkan kasus ini jauh lebih pelik.

\hyphenation{ber-ko-re-la-si pe-la-ku-nya atau-pun}

Selain karena sebuah pesan singkat \textit{aku akan bunuh diri, jangan tolong aku} terdengar seperti gurauan belaka, tidak ada seorang pun dapat menyingkap isi benak orang lain. Bahkan sekalipun seseorang berterus terang, kurasa seorang pendengar yang baik tidak akan mampu untuk melangkah lebih jauh dari sekadar menerka-nerka maksudnya belaka; dia tetap tidak akan benar-benar memahaminya. Kecuali orang-orang yang suka mengonfirmasi setiap kalimat yang dia dengar, yah, sedikit sekali yang seperti itu `kan. Tiga atau empat kalimat saja tidak akan cukup untuk menyimpulkan bahwa seseorang akan bunuh diri sore hari ini. Dibutuhkan semacam pengenalan pola dari tulisan-tulisan sebelumnya, baik dari tulisan yang bersangkutan itu sendiri maupun tulisan orang lain mengenai dirinya. Sehingga prakiraan acak berkenaan dengan kapan dan di mana seseorang akan bunuh diri dapat dibuat. Nyatanya, hanya menemukan beberapa kalimat yang berkorelasi sudah merupakan karunia, sebab umumnya pelaku bunuh diri akan menutup diri. Pada akhirnya tiada seorang pun yang dapat bertindak sampai tubuhnya ditemukan tetangganya telah mengejur dalam kesepian yang teramat mendalam.

\separator{}

Siang ini aku melaporkan peristiwa malam kemarin kepada atasanku. Dia berbalik untuk mendengarkanku, namun tetap tidak menaruh perhatian sedikit pun. Tidak banyak yang bisa kukerjakan sekembalinya ke ruang tidurku. Aku menatap langit-langit untuk dua puluh menit ke depan, percis seperti yang kulakukan empat tahun yang lalu. Kematian itu datang terlalu mendadak. Andai saja ia mencegatku dari pergi berlayar dua hari sebelumnya. Andai saja aku tidak pernah ragu untuk menghadiri undangan makan malam berdua dengan Isha. Andai saja aku berhasil menghadapi rasa takutku terhadap Egg. Andai saja aku sanggup mengikuti kata-kataku sendiri. Namun segalanya terus-menerus terlambat, padahal masa pernah beberapa kali melambat.

\hyphenation{meng-ge-lin-ding-kan}

Pukul empat sore hari itu, aku melihat seorang wanita yang sudah baya turun dari sepedanya kira-kira sepuluh langkah dari tempatku duduk. Dia melambai-lambaikan sepucuk amplop jingga ke arahku. Aku ingin sekali menggelindingkan kursi rodaku secepat mungkin untuk menjemputnya, namun dia sudah telanjur memasukkannya ke dalam kotak surat dan pergi menenteng sepedanya ke arah dia tadi datang.

\hyphenation{wa-ni-ta ber-a-ngin}

Dua bulan sejak kecelakaanku di kamp pelatihan, wanita itu tidak lagi bersikeras memintaku mendengarkan sedikit cerita tentang orang-orang yang sudah menungguinya sepanjang hari. Aku memang pernah menyarankannya untuk bercerita ketika mendapati dirinya dicemooh selagi bertugas. Namun, aku tidak menyangka bahwa dia tidak pernah bisa berhenti sejak itu. Pada dasarnya, dia memang orang yang senang bercerita. Tidak membosankan, meskipun cerita dan tokoh-tokohnya hampir tidak pernah berubah. Mungkin bukan ceritanya yang menarik, tapi gayanya bercerita. Dia selalu mengingatkanku pada hari ketika aku terpaksa menginap dan mendengarkan Eri, neneknya Egg, membawakan sebuah dongeng.

\hyphenation{ta-ngan}

Kedatangannya sungguh mengobati kesepianku. Dua hal yang aku senangi dari kedatangannya: cerita tentang orang-orang dan surat dari Isha. Ya, barang tentu itu darinya. Aku bisa merasakan perasaan hangat dan damai mengalir ke seluruh nadiku. Aku tidak mengira itu akan sebegitunya, mengingat itu menjadi tulisan tangan terakhir darinya. Maka jadilah aku orang yang paling menyesal dalam kebahagiaan yang singkat itu.

\separator{}

\hyphenation{se-ka-li}

Aku datang pukul sepuluh lebih empat puluh dua detik. Isha yang biasanya duduk di sisi sebelah sana, kini terduduk di kursi pesakitan dalam kegamangan. Kantung matanya menghitam dan bengkak, wajahnya sepucat porselen, ia tidak bisa tidur selama tiga hari terakhir. Dadaku ikut sesak, napasku tersengal-sengal, mataku perih melihat kondisinya yang begitu lemah. Dia bahkan tidak mampu bangkit dari kursi itu sendirian.

Hakim memutuskan secara sepihak bahwa tersangka akan dibiarkan hidup dalam sel selama empat bulan sebelum menjalani hukuman mati. Tidak ada seorang pun yang menyuarakan keberatan, sebab semua orang tahu benar bahwa itu semua hanyalah sebuah sandiwara satu babak yang ditulis oleh pihak penggugat. Setiap orang yang berusaha merebut kemudi dapat dipastikan akan menghilang pada pagi berikutnya.

Hari itu bagaikan neraka bagi kami berenam. Waktu benar-benar melambat selama itu, sementara kami terus berjuang menahan mual mendengar ucapan bombastis si penggungat dan melihat sikap sinis sang hakim. Lalu tiba-tiba waktu berlari-lari dengan seenaknya, meninggalkan kami yang membalap di belakang jatuh tersungkur sambil memuntahkan semua isi perut. Tidak ada lagi jalan di depan, semuanya sudah terlambat, lalu kami pun berbalik pulang. Mimpi yang sangat buruk, andai saja begitu.

Aku dan Egg mengingatnya dengan jelas, air mukanya tidak pernah berubah sama sekali ketika kami kembali melihatnya dan untuk terakhir kalinya sejak hari itu. Padahal selama kami mengenalnya, wajahnya selalu tampak ceria dan hidup walaupun tidak harus selalu tersenyum, bahkan sebetulnya dia termasuk yang jarang senyum. Namun bagaimanapun juga, dia tetap saja cantik, meski terbaring di dalam peti mati dari kayu jati. Dan akan selalu cantik, meski tanpa sehelai rambut pun tersisa di kepalanya. Hanya aku dan kemudian Egg yang awalnya tahu bahwa sebulan lalu Isha mulai mengenakan rambut tiruan.

\hyphenation{ha-nya}

Bukan dengan sebab rambutnya itu aku tidak menjawab undangannya, bukan juga itu bagi Egg. Kami tetap akan menyayanginya apapun yang terjadi. Namun Isha sendiri yang meminta untuk berhenti memikirkannya dan sebagai gantinya meminta kami berdua untuk berbaikan pada diri kami masing-masing. Dia benar-benar tulus menghibur kami yang saat itu sedang pilu bagai diiris sembilu, bermuram durja tanpa daya, menyalahi diri sendiri.

\separator{}

\hyphenation{meng-a-ki-bat-kan}

Para anggota lapas dan penjaga rutan lontang-lantung histeris. Para tahanan yang baru saja dikeluarkan dalam sekejap sudah berada di dalam truk-truk pengangkut prajurit. Pasalnya air bah sudah mencapai lutut hanya dalam lima atau enam menit. Sore berkabut itu hujan turun sangat deras, namun seharusnya tidak akan pernah mengakibatkan banjir. Sehingga peristiwa tersebut sudah pasti kami berlima yang harus disalahkan.

\hyphenation{ke-hi-lang-an meng-a-lir-kan me-min-ta-ku}

Ara, si pemilik ide untuk menjaga agar selokan yang besar itu kehilangan fungsinya selama sepuluh menit, memintaku sebagai pemilik awak yang paling tegap kuat untuk membantunya memindahkan karung-karung pasir yang berat itu guna mencegat air comberan itu terus mengalir menuju ke hilir sungai. Selokan itu memang besar, namun tidak dapat dipungkiri bahwa ia merupakan titik temu bagi banyak selokan-selokan kecil lainnya. Pada jam-jam teratur, selokan-selokan kecil itu mengalirkan air dalam debit beragam secara bergantian. Tepat pada jam lima sore, selokan-selokan kecil itu bersamaan akan mengalirkan air limbah yang begitu deras. Tidak ada yang tahu selama ini tentang selokan besar itu yang selalu kewalahan pada jam-jam tersebut, sebab di kala itu orang-orang baru saja hendak meninggalkan pabrik-pabrik di mana mereka membutuhkan waktu sekitar sepuluh menit untuk sampai pada jalan yang hanya bisa dilewati per dua orang itu. Sementara selokan besar itu sudah akan kembali menganggur hanya dalam kurun waktu kurang dari lima menit.

Ditambah lagi hujannya sangat deras, membuka lebih banyak kesempatan untuk menyelamatkan Isha keluar dari sel tahanan. Sementara itu, kami semua tahu bahwa Isha tidak akan menyukainya sama sekali, meskipun dia tidak akan sampai hati membenci kami.

\hyphenation{bu-kan-lah}

Isha sering menasihatiku, ``Lee-ku, hidup itu harus selalu dihadapi dengan keyakinan. Yakinkanlah dirimu bahwa kebenaran akan selalu menang apapun yang terjadi, meskipun terkadang untuk menang butuh pengorbanan yang banyak.'' Namun, hari itu, kuyakinkan diriku bahwa Isha telah salah dalam ucapannya yang selalu kuingin percayai sepanjang hidupku. Bahwa kehilangan dirinya bukanlah pengorbanan besar yang patut dipertimbangkan demi memenangkan sebuah kebenaran. Maka tidak ada yang meragukan ide tersebut di antara kami berlima sejak itu pertama kali direncanakan.

Semuanya berkumpul mengawasi di antara pepohonan dan semak belukar yang tumbuh di seberang rutan. Para tahanan satu per satu diangkut ke dalam truk-truk. Namun hingga menit-menit terakhir truk-truk tersebut meluncur pergi melarikan para tahanan ke tempat yang lebih tinggi, Isha tetap tidak terlihat sama sekali berjalan keluar melewati pintu depan rutan. Kami merasa sangat bodoh, apakah kami semua telah melewatkannya atau Isha memang tidak pernah keluar dari sana. Kami saling memandang satu sama lain. Namun saking kalang kabutnya, tidak ada yang menyadari seorang pun dari kami bahwa Ara telah menghilang dari tempat itu entah sejak kapan.

Tidak ada yang akan mengingatnya jika saja Isha tidak pernah menceritakannya, bahwa Ara bertindak sangat gegabah pada sore itu. Hanya itu yang Isha katakan, tidak lebih juga tidak kurang. Dia tetap bersikeras merahasiakan apa yang sebenarnya terjadi, bahkan dari para dokter yang merawat Ara sekalipun. Katanya, Ara yang menginginkannya sendiri.

\hyphenation{sek-re-ta-ris-nya}

Misteri itu belum juga terungkap dan mungkin tetap akan seperti itu. Ara masih berbaring koma di salah satu kamar ICU tidak berdaya pada hari di mana Isha ditemukan dalam direnggut nyawanya di kediaman bekas sekretarisnya oleh salah seorang tetangga yang cerewet.

\separator{}

\hyphenation{teng-ko-rak-ku ku-ce-ri-ta-kan ka-ta-nya}

Aku pernah bermimpi dalam tidurku yang sedikit lebih nyenyak daripada biasanya, sebuah mimpi yang kuceritakan kepada Isha keesokan harinya dalam kunjungan pekananku. Dia nyaris terbahak-bahak. Katanya tengkorakku terlalu keras untuk bisa terjatuh dari kapal sehingga menderita amnesia akibat benturan batu karang di dasar laut. Dia menjelaskan bahwa batu karang itulah yang sepatutnya menderita amnesia akibat sundulanku.

``Kasihan,'' ujarnya, ``oh batu karang yang sedang tidak beruntung.''

\hyphenation{ingat-lah}

``Seandainya pun batu karang itu sudah cukup berpengalaman dalam menghindari serangan semacam itu, ingatlah selalu bahwa sepandai-pandai tupat melompat, sekali waktu gawal juga,'' balasku.

Aku dan Isha sudah berteman baik sejak kami masih dalam buaian. Dia lahir dua bulan lebih dahulu. Dia bukan tetanggaku, melainkan kedua orang tuanya adalah mitra dagang ayahku sehingga kami saling bertemu setidaknya empat kali setahun, sampai ibuku memperkenalkannya kepada pamanku. Maka keluargaku membantu Isha pindah. Isha awalnya sangat-sangat menentang keputusan sepihak itu. Namun, pamanku berhasil membujuk Isha untuk menyukai dirinya.

\hyphenation{ber-war-na}

Nahas ayah Isha disusul ibunya dinyatakan hilang menjelang krisis politik di Bangladesh pada akhir tahun 2006. Namun baru-baru ini diketahui bahwa kedua orang tuanya masih hidup dalam keadaan sehat walafiat hingga hari ini. Yakni dari sepucuk surat di dalam sebuah kotak sepatu perempuan bersamaan dengan sebuah bebek kayu berwarna polos. Kotak itu diantarkan ke depan pintu rumahku oleh seseorang yang mengenakan jas hujan berwarna gelap pada pukul dua dini hari, beberapa saat sebelum Isha menghembuskan napas terakhirnya.

\hyphenation{ber-sau-da-ra}

Surat itu diawali dengan sepenggal kisah mengerikan tentang salah seorang anak perempuan kembar dari dua bersaudara yang meninggal di usia empat tahun akibat dibunuh dengan keji oleh---orang-orang yang disebut oleh para polisi setempat sebagai---para perampok kawakan.

% Egg, Lee, Asse, Ara, dan Karin menerima kisah yang sama, namun dengan penggalan yang berbeda.

\chapter{Pensil Inul}

``Kedai kopi Java Cup pukul sembilan. Aku akan datang dengan sweter hijau gelap.''

Lee langsung menutup telepon tanpa memberikanku kesempatan bertanya mengenai apa yang harus kubawa. Maka segera kusakukan salah satu surya kanta terbaikku dan bergegas mencari taksi.

Butuh sekitar empat puluh menit untuk sampai ke sana. Sementara Lee bisa dengan mudah berlari kecil selama sepuluh menit untuk bolak-balik dari rumahnya ke kedai tersebut jika diperlukan. Aku tidak mempermasalahkan itu sama sekali, mengingat ini sudah menjadi yang kesekian kalinya untukku. Aku sudah terbiasa. Kecuali Lee terdengar sangat gelisah, ini pertama kalinya. Kiranya hal apa yang membebani seorang bekas perwira laut yang terkenal sangat tegar?

\hyphenation{mem-ba-ngun-kan-ku ber-pi-kir ku-ki-ra me-nge-li-li-ngi}

Si supir taksi memanggil-manggil dari jok depan mencoba membangunkanku. Aku tersenyum malu-malu sambil mengusap-usap mataku yang sedikit berair. Berpikir terlalu keras membuat otakku cepat sekali lelah. Kukira inilah saatnya untuk memesan asupan gula ekstra. Buru-buru aku membayar lalu turun dari taksi. Aku mendongak ke atas sementara taksi itu memutar balik lalu menyeruak pergi ke dalam kabut malam yang tipis. Huruf-huruf cetak di papan nama kedai itu sudah mulai kabur tersapu air hujan. Tidak seperti kedai-kedai lainnya, papan namanya benar-benar biasa sekali; tidak ada lampu-lampu neon yang mengelilingi papan tersebut. Debu-debu yang singgah dari kehidupan kota yang panjang bertambah banyak menutupi papan nama tersebut.

``Asse, di sini!''

% \chapter{Sepatu Kets}

% Opera ini, aku khawatir, tidak akan berjalan semulus itu. Isha sering memperdengarkanku kalimat kegemarannya. \textit{Asal ayam pulang ke lumbung, asal itik pulang ke pelimbahan.} Tidak jelas siapa yang menyamar menjadi siapa, bahkan aku sendiri tidak tahu aku sedang berperan menjadi siapa. Apakah sebagai Heri, sekretaris kepercayaan Isha, atau sebagai Egg yang selama ini dikenal baik oleh Isha? Terlepas dari itu, aku harus memposisikan diri sebagai Isha untuk menceritakan kejadian yang sebenarnya.

% Malam itu sekitar pukul delapan, Lee menjumpai seorang paman di balik pintu putar di lobi utama tempatnya menginap selama lima hari ke depan, katanya memulai cerita. Eksentrik adalah kata pertama yang terpikirkan olehnya. Paman itu mengenakan setelan jas mewah, dari mulai rambut hingga sepatunya serba putih. Kecuali begitu, dasinya berwarna gelap menyimpul ala Eldredge. Lee keasyikan memandang simpul itu sehingga lupa sejenak kalau yang berdiri di hadapannya itu bukan model orang-orangan yang dipajang di etalase depan toko busana.

% ``Benar, begitulah seharusnya orang-orang memandang satu sama lain. ''

% akhir cerita, lalu mundur dan maju secara bertahap sampai bagian di mana Heri alias Egg yang sedang menyamar masuk ke ruangan

% Egg mengirimiku sebuah buku kosong bertajuk \textit{Seseorang Berangkat Menuju Surga} dengan guntingan-guntingan pers yang kutemukan terselip di dalamnya. Entah mengapa, aku merasa familiar dengan tulisan tangannya ...

% Aku membacanya keras-keras aksara sendu yang diselipkan di bawah daun pintu kamarku oleh si pertapa gila, ``Semboyan yang muluk-muluk itu tidak banyak gunanya, lebih baik bertindak.''

% Gula batu dengan daging asap, benar-benar tidak cocok sekali. Ha-ha-ha ...

% \chapter{Hantu Trampolin}

% Memalukan, sebetulnya Lee lebih memahami Isha dibandingkan diriku.

% Kami tahu, ini akan menjadi pukulan berat bagi Karin \dots

% ``Semua kita tahu bahwa Isha baik kepada semua orang, tetapi \dots''

% ``Aku tahu betul, cokelat ini akan menjadi sebab terbunuhnya diriku. Namun kumakan langsung tanpa ragu ... untuk Ara.''

% ``Kecuali dirimu telah duduk di sampingnya, Isha tidak pernah tertawa lepas di hadapan teman-temannya, sekalipun bukan dikau yang sedang berkelakar.''

% Peti dari kayu jati itu sangat mahal. Lee dan Egg pun harus berpatungan untuk membelikannya.

% begin back matter

\end{document}
% END THE DOCUMENT
